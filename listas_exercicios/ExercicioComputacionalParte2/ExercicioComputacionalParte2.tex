\documentclass[11pt]{article}

    \usepackage[breakable]{tcolorbox}
    \usepackage{parskip} % Stop auto-indenting (to mimic markdown behaviour)
    
    \usepackage{iftex}
    \ifPDFTeX
    	\usepackage[T1]{fontenc}
    	\usepackage{mathpazo}
    \else
    	\usepackage{fontspec}
    \fi

    % Basic figure setup, for now with no caption control since it's done
    % automatically by Pandoc (which extracts ![](path) syntax from Markdown).
    \usepackage{graphicx}
    % Maintain compatibility with old templates. Remove in nbconvert 6.0
    \let\Oldincludegraphics\includegraphics
    % Ensure that by default, figures have no caption (until we provide a
    % proper Figure object with a Caption API and a way to capture that
    % in the conversion process - todo).
    \usepackage{caption}
    \DeclareCaptionFormat{nocaption}{}
    \captionsetup{format=nocaption,aboveskip=0pt,belowskip=0pt}

    \usepackage{float}
    \floatplacement{figure}{H} % forces figures to be placed at the correct location
    \usepackage{xcolor} % Allow colors to be defined
    \usepackage{enumerate} % Needed for markdown enumerations to work
    \usepackage{geometry} % Used to adjust the document margins
    \usepackage{amsmath} % Equations
    \usepackage{amssymb} % Equations
    \usepackage{textcomp} % defines textquotesingle
    % Hack from http://tex.stackexchange.com/a/47451/13684:
    \AtBeginDocument{%
        \def\PYZsq{\textquotesingle}% Upright quotes in Pygmentized code
    }
    \usepackage{upquote} % Upright quotes for verbatim code
    \usepackage{eurosym} % defines \euro
    \usepackage[mathletters]{ucs} % Extended unicode (utf-8) support
    \usepackage{fancyvrb} % verbatim replacement that allows latex
    \usepackage{grffile} % extends the file name processing of package graphics 
                         % to support a larger range
    \makeatletter % fix for old versions of grffile with XeLaTeX
    \@ifpackagelater{grffile}{2019/11/01}
    {
      % Do nothing on new versions
    }
    {
      \def\Gread@@xetex#1{%
        \IfFileExists{"\Gin@base".bb}%
        {\Gread@eps{\Gin@base.bb}}%
        {\Gread@@xetex@aux#1}%
      }
    }
    \makeatother
    \usepackage[Export]{adjustbox} % Used to constrain images to a maximum size
    \adjustboxset{max size={0.9\linewidth}{0.9\paperheight}}

    % The hyperref package gives us a pdf with properly built
    % internal navigation ('pdf bookmarks' for the table of contents,
    % internal cross-reference links, web links for URLs, etc.)
    \usepackage{hyperref}
    % The default LaTeX title has an obnoxious amount of whitespace. By default,
    % titling removes some of it. It also provides customization options.
    \usepackage{titling}
    \usepackage{longtable} % longtable support required by pandoc >1.10
    \usepackage{booktabs}  % table support for pandoc > 1.12.2
    \usepackage[inline]{enumitem} % IRkernel/repr support (it uses the enumerate* environment)
    \usepackage[normalem]{ulem} % ulem is needed to support strikethroughs (\sout)
                                % normalem makes italics be italics, not underlines
    \usepackage{mathrsfs}
    

    
    % Colors for the hyperref package
    \definecolor{urlcolor}{rgb}{0,.145,.698}
    \definecolor{linkcolor}{rgb}{.71,0.21,0.01}
    \definecolor{citecolor}{rgb}{.12,.54,.11}

    % ANSI colors
    \definecolor{ansi-black}{HTML}{3E424D}
    \definecolor{ansi-black-intense}{HTML}{282C36}
    \definecolor{ansi-red}{HTML}{E75C58}
    \definecolor{ansi-red-intense}{HTML}{B22B31}
    \definecolor{ansi-green}{HTML}{00A250}
    \definecolor{ansi-green-intense}{HTML}{007427}
    \definecolor{ansi-yellow}{HTML}{DDB62B}
    \definecolor{ansi-yellow-intense}{HTML}{B27D12}
    \definecolor{ansi-blue}{HTML}{208FFB}
    \definecolor{ansi-blue-intense}{HTML}{0065CA}
    \definecolor{ansi-magenta}{HTML}{D160C4}
    \definecolor{ansi-magenta-intense}{HTML}{A03196}
    \definecolor{ansi-cyan}{HTML}{60C6C8}
    \definecolor{ansi-cyan-intense}{HTML}{258F8F}
    \definecolor{ansi-white}{HTML}{C5C1B4}
    \definecolor{ansi-white-intense}{HTML}{A1A6B2}
    \definecolor{ansi-default-inverse-fg}{HTML}{FFFFFF}
    \definecolor{ansi-default-inverse-bg}{HTML}{000000}

    % common color for the border for error outputs.
    \definecolor{outerrorbackground}{HTML}{FFDFDF}

    % commands and environments needed by pandoc snippets
    % extracted from the output of `pandoc -s`
    \providecommand{\tightlist}{%
      \setlength{\itemsep}{0pt}\setlength{\parskip}{0pt}}
    \DefineVerbatimEnvironment{Highlighting}{Verbatim}{commandchars=\\\{\}}
    % Add ',fontsize=\small' for more characters per line
    \newenvironment{Shaded}{}{}
    \newcommand{\KeywordTok}[1]{\textcolor[rgb]{0.00,0.44,0.13}{\textbf{{#1}}}}
    \newcommand{\DataTypeTok}[1]{\textcolor[rgb]{0.56,0.13,0.00}{{#1}}}
    \newcommand{\DecValTok}[1]{\textcolor[rgb]{0.25,0.63,0.44}{{#1}}}
    \newcommand{\BaseNTok}[1]{\textcolor[rgb]{0.25,0.63,0.44}{{#1}}}
    \newcommand{\FloatTok}[1]{\textcolor[rgb]{0.25,0.63,0.44}{{#1}}}
    \newcommand{\CharTok}[1]{\textcolor[rgb]{0.25,0.44,0.63}{{#1}}}
    \newcommand{\StringTok}[1]{\textcolor[rgb]{0.25,0.44,0.63}{{#1}}}
    \newcommand{\CommentTok}[1]{\textcolor[rgb]{0.38,0.63,0.69}{\textit{{#1}}}}
    \newcommand{\OtherTok}[1]{\textcolor[rgb]{0.00,0.44,0.13}{{#1}}}
    \newcommand{\AlertTok}[1]{\textcolor[rgb]{1.00,0.00,0.00}{\textbf{{#1}}}}
    \newcommand{\FunctionTok}[1]{\textcolor[rgb]{0.02,0.16,0.49}{{#1}}}
    \newcommand{\RegionMarkerTok}[1]{{#1}}
    \newcommand{\ErrorTok}[1]{\textcolor[rgb]{1.00,0.00,0.00}{\textbf{{#1}}}}
    \newcommand{\NormalTok}[1]{{#1}}
    
    % Additional commands for more recent versions of Pandoc
    \newcommand{\ConstantTok}[1]{\textcolor[rgb]{0.53,0.00,0.00}{{#1}}}
    \newcommand{\SpecialCharTok}[1]{\textcolor[rgb]{0.25,0.44,0.63}{{#1}}}
    \newcommand{\VerbatimStringTok}[1]{\textcolor[rgb]{0.25,0.44,0.63}{{#1}}}
    \newcommand{\SpecialStringTok}[1]{\textcolor[rgb]{0.73,0.40,0.53}{{#1}}}
    \newcommand{\ImportTok}[1]{{#1}}
    \newcommand{\DocumentationTok}[1]{\textcolor[rgb]{0.73,0.13,0.13}{\textit{{#1}}}}
    \newcommand{\AnnotationTok}[1]{\textcolor[rgb]{0.38,0.63,0.69}{\textbf{\textit{{#1}}}}}
    \newcommand{\CommentVarTok}[1]{\textcolor[rgb]{0.38,0.63,0.69}{\textbf{\textit{{#1}}}}}
    \newcommand{\VariableTok}[1]{\textcolor[rgb]{0.10,0.09,0.49}{{#1}}}
    \newcommand{\ControlFlowTok}[1]{\textcolor[rgb]{0.00,0.44,0.13}{\textbf{{#1}}}}
    \newcommand{\OperatorTok}[1]{\textcolor[rgb]{0.40,0.40,0.40}{{#1}}}
    \newcommand{\BuiltInTok}[1]{{#1}}
    \newcommand{\ExtensionTok}[1]{{#1}}
    \newcommand{\PreprocessorTok}[1]{\textcolor[rgb]{0.74,0.48,0.00}{{#1}}}
    \newcommand{\AttributeTok}[1]{\textcolor[rgb]{0.49,0.56,0.16}{{#1}}}
    \newcommand{\InformationTok}[1]{\textcolor[rgb]{0.38,0.63,0.69}{\textbf{\textit{{#1}}}}}
    \newcommand{\WarningTok}[1]{\textcolor[rgb]{0.38,0.63,0.69}{\textbf{\textit{{#1}}}}}
    
    
    % Define a nice break command that doesn't care if a line doesn't already
    % exist.
    \def\br{\hspace*{\fill} \\* }
    % Math Jax compatibility definitions
    \def\gt{>}
    \def\lt{<}
    \let\Oldtex\TeX
    \let\Oldlatex\LaTeX
    \renewcommand{\TeX}{\textrm{\Oldtex}}
    \renewcommand{\LaTeX}{\textrm{\Oldlatex}}
    % Document parameters
    % Document title
    \title{ExercicioComputacionalParte2}
    
    
    
    
    
% Pygments definitions
\makeatletter
\def\PY@reset{\let\PY@it=\relax \let\PY@bf=\relax%
    \let\PY@ul=\relax \let\PY@tc=\relax%
    \let\PY@bc=\relax \let\PY@ff=\relax}
\def\PY@tok#1{\csname PY@tok@#1\endcsname}
\def\PY@toks#1+{\ifx\relax#1\empty\else%
    \PY@tok{#1}\expandafter\PY@toks\fi}
\def\PY@do#1{\PY@bc{\PY@tc{\PY@ul{%
    \PY@it{\PY@bf{\PY@ff{#1}}}}}}}
\def\PY#1#2{\PY@reset\PY@toks#1+\relax+\PY@do{#2}}

\expandafter\def\csname PY@tok@w\endcsname{\def\PY@tc##1{\textcolor[rgb]{0.73,0.73,0.73}{##1}}}
\expandafter\def\csname PY@tok@c\endcsname{\let\PY@it=\textit\def\PY@tc##1{\textcolor[rgb]{0.25,0.50,0.50}{##1}}}
\expandafter\def\csname PY@tok@cp\endcsname{\def\PY@tc##1{\textcolor[rgb]{0.74,0.48,0.00}{##1}}}
\expandafter\def\csname PY@tok@k\endcsname{\let\PY@bf=\textbf\def\PY@tc##1{\textcolor[rgb]{0.00,0.50,0.00}{##1}}}
\expandafter\def\csname PY@tok@kp\endcsname{\def\PY@tc##1{\textcolor[rgb]{0.00,0.50,0.00}{##1}}}
\expandafter\def\csname PY@tok@kt\endcsname{\def\PY@tc##1{\textcolor[rgb]{0.69,0.00,0.25}{##1}}}
\expandafter\def\csname PY@tok@o\endcsname{\def\PY@tc##1{\textcolor[rgb]{0.40,0.40,0.40}{##1}}}
\expandafter\def\csname PY@tok@ow\endcsname{\let\PY@bf=\textbf\def\PY@tc##1{\textcolor[rgb]{0.67,0.13,1.00}{##1}}}
\expandafter\def\csname PY@tok@nb\endcsname{\def\PY@tc##1{\textcolor[rgb]{0.00,0.50,0.00}{##1}}}
\expandafter\def\csname PY@tok@nf\endcsname{\def\PY@tc##1{\textcolor[rgb]{0.00,0.00,1.00}{##1}}}
\expandafter\def\csname PY@tok@nc\endcsname{\let\PY@bf=\textbf\def\PY@tc##1{\textcolor[rgb]{0.00,0.00,1.00}{##1}}}
\expandafter\def\csname PY@tok@nn\endcsname{\let\PY@bf=\textbf\def\PY@tc##1{\textcolor[rgb]{0.00,0.00,1.00}{##1}}}
\expandafter\def\csname PY@tok@ne\endcsname{\let\PY@bf=\textbf\def\PY@tc##1{\textcolor[rgb]{0.82,0.25,0.23}{##1}}}
\expandafter\def\csname PY@tok@nv\endcsname{\def\PY@tc##1{\textcolor[rgb]{0.10,0.09,0.49}{##1}}}
\expandafter\def\csname PY@tok@no\endcsname{\def\PY@tc##1{\textcolor[rgb]{0.53,0.00,0.00}{##1}}}
\expandafter\def\csname PY@tok@nl\endcsname{\def\PY@tc##1{\textcolor[rgb]{0.63,0.63,0.00}{##1}}}
\expandafter\def\csname PY@tok@ni\endcsname{\let\PY@bf=\textbf\def\PY@tc##1{\textcolor[rgb]{0.60,0.60,0.60}{##1}}}
\expandafter\def\csname PY@tok@na\endcsname{\def\PY@tc##1{\textcolor[rgb]{0.49,0.56,0.16}{##1}}}
\expandafter\def\csname PY@tok@nt\endcsname{\let\PY@bf=\textbf\def\PY@tc##1{\textcolor[rgb]{0.00,0.50,0.00}{##1}}}
\expandafter\def\csname PY@tok@nd\endcsname{\def\PY@tc##1{\textcolor[rgb]{0.67,0.13,1.00}{##1}}}
\expandafter\def\csname PY@tok@s\endcsname{\def\PY@tc##1{\textcolor[rgb]{0.73,0.13,0.13}{##1}}}
\expandafter\def\csname PY@tok@sd\endcsname{\let\PY@it=\textit\def\PY@tc##1{\textcolor[rgb]{0.73,0.13,0.13}{##1}}}
\expandafter\def\csname PY@tok@si\endcsname{\let\PY@bf=\textbf\def\PY@tc##1{\textcolor[rgb]{0.73,0.40,0.53}{##1}}}
\expandafter\def\csname PY@tok@se\endcsname{\let\PY@bf=\textbf\def\PY@tc##1{\textcolor[rgb]{0.73,0.40,0.13}{##1}}}
\expandafter\def\csname PY@tok@sr\endcsname{\def\PY@tc##1{\textcolor[rgb]{0.73,0.40,0.53}{##1}}}
\expandafter\def\csname PY@tok@ss\endcsname{\def\PY@tc##1{\textcolor[rgb]{0.10,0.09,0.49}{##1}}}
\expandafter\def\csname PY@tok@sx\endcsname{\def\PY@tc##1{\textcolor[rgb]{0.00,0.50,0.00}{##1}}}
\expandafter\def\csname PY@tok@m\endcsname{\def\PY@tc##1{\textcolor[rgb]{0.40,0.40,0.40}{##1}}}
\expandafter\def\csname PY@tok@gh\endcsname{\let\PY@bf=\textbf\def\PY@tc##1{\textcolor[rgb]{0.00,0.00,0.50}{##1}}}
\expandafter\def\csname PY@tok@gu\endcsname{\let\PY@bf=\textbf\def\PY@tc##1{\textcolor[rgb]{0.50,0.00,0.50}{##1}}}
\expandafter\def\csname PY@tok@gd\endcsname{\def\PY@tc##1{\textcolor[rgb]{0.63,0.00,0.00}{##1}}}
\expandafter\def\csname PY@tok@gi\endcsname{\def\PY@tc##1{\textcolor[rgb]{0.00,0.63,0.00}{##1}}}
\expandafter\def\csname PY@tok@gr\endcsname{\def\PY@tc##1{\textcolor[rgb]{1.00,0.00,0.00}{##1}}}
\expandafter\def\csname PY@tok@ge\endcsname{\let\PY@it=\textit}
\expandafter\def\csname PY@tok@gs\endcsname{\let\PY@bf=\textbf}
\expandafter\def\csname PY@tok@gp\endcsname{\let\PY@bf=\textbf\def\PY@tc##1{\textcolor[rgb]{0.00,0.00,0.50}{##1}}}
\expandafter\def\csname PY@tok@go\endcsname{\def\PY@tc##1{\textcolor[rgb]{0.53,0.53,0.53}{##1}}}
\expandafter\def\csname PY@tok@gt\endcsname{\def\PY@tc##1{\textcolor[rgb]{0.00,0.27,0.87}{##1}}}
\expandafter\def\csname PY@tok@err\endcsname{\def\PY@bc##1{\setlength{\fboxsep}{0pt}\fcolorbox[rgb]{1.00,0.00,0.00}{1,1,1}{\strut ##1}}}
\expandafter\def\csname PY@tok@kc\endcsname{\let\PY@bf=\textbf\def\PY@tc##1{\textcolor[rgb]{0.00,0.50,0.00}{##1}}}
\expandafter\def\csname PY@tok@kd\endcsname{\let\PY@bf=\textbf\def\PY@tc##1{\textcolor[rgb]{0.00,0.50,0.00}{##1}}}
\expandafter\def\csname PY@tok@kn\endcsname{\let\PY@bf=\textbf\def\PY@tc##1{\textcolor[rgb]{0.00,0.50,0.00}{##1}}}
\expandafter\def\csname PY@tok@kr\endcsname{\let\PY@bf=\textbf\def\PY@tc##1{\textcolor[rgb]{0.00,0.50,0.00}{##1}}}
\expandafter\def\csname PY@tok@bp\endcsname{\def\PY@tc##1{\textcolor[rgb]{0.00,0.50,0.00}{##1}}}
\expandafter\def\csname PY@tok@fm\endcsname{\def\PY@tc##1{\textcolor[rgb]{0.00,0.00,1.00}{##1}}}
\expandafter\def\csname PY@tok@vc\endcsname{\def\PY@tc##1{\textcolor[rgb]{0.10,0.09,0.49}{##1}}}
\expandafter\def\csname PY@tok@vg\endcsname{\def\PY@tc##1{\textcolor[rgb]{0.10,0.09,0.49}{##1}}}
\expandafter\def\csname PY@tok@vi\endcsname{\def\PY@tc##1{\textcolor[rgb]{0.10,0.09,0.49}{##1}}}
\expandafter\def\csname PY@tok@vm\endcsname{\def\PY@tc##1{\textcolor[rgb]{0.10,0.09,0.49}{##1}}}
\expandafter\def\csname PY@tok@sa\endcsname{\def\PY@tc##1{\textcolor[rgb]{0.73,0.13,0.13}{##1}}}
\expandafter\def\csname PY@tok@sb\endcsname{\def\PY@tc##1{\textcolor[rgb]{0.73,0.13,0.13}{##1}}}
\expandafter\def\csname PY@tok@sc\endcsname{\def\PY@tc##1{\textcolor[rgb]{0.73,0.13,0.13}{##1}}}
\expandafter\def\csname PY@tok@dl\endcsname{\def\PY@tc##1{\textcolor[rgb]{0.73,0.13,0.13}{##1}}}
\expandafter\def\csname PY@tok@s2\endcsname{\def\PY@tc##1{\textcolor[rgb]{0.73,0.13,0.13}{##1}}}
\expandafter\def\csname PY@tok@sh\endcsname{\def\PY@tc##1{\textcolor[rgb]{0.73,0.13,0.13}{##1}}}
\expandafter\def\csname PY@tok@s1\endcsname{\def\PY@tc##1{\textcolor[rgb]{0.73,0.13,0.13}{##1}}}
\expandafter\def\csname PY@tok@mb\endcsname{\def\PY@tc##1{\textcolor[rgb]{0.40,0.40,0.40}{##1}}}
\expandafter\def\csname PY@tok@mf\endcsname{\def\PY@tc##1{\textcolor[rgb]{0.40,0.40,0.40}{##1}}}
\expandafter\def\csname PY@tok@mh\endcsname{\def\PY@tc##1{\textcolor[rgb]{0.40,0.40,0.40}{##1}}}
\expandafter\def\csname PY@tok@mi\endcsname{\def\PY@tc##1{\textcolor[rgb]{0.40,0.40,0.40}{##1}}}
\expandafter\def\csname PY@tok@il\endcsname{\def\PY@tc##1{\textcolor[rgb]{0.40,0.40,0.40}{##1}}}
\expandafter\def\csname PY@tok@mo\endcsname{\def\PY@tc##1{\textcolor[rgb]{0.40,0.40,0.40}{##1}}}
\expandafter\def\csname PY@tok@ch\endcsname{\let\PY@it=\textit\def\PY@tc##1{\textcolor[rgb]{0.25,0.50,0.50}{##1}}}
\expandafter\def\csname PY@tok@cm\endcsname{\let\PY@it=\textit\def\PY@tc##1{\textcolor[rgb]{0.25,0.50,0.50}{##1}}}
\expandafter\def\csname PY@tok@cpf\endcsname{\let\PY@it=\textit\def\PY@tc##1{\textcolor[rgb]{0.25,0.50,0.50}{##1}}}
\expandafter\def\csname PY@tok@c1\endcsname{\let\PY@it=\textit\def\PY@tc##1{\textcolor[rgb]{0.25,0.50,0.50}{##1}}}
\expandafter\def\csname PY@tok@cs\endcsname{\let\PY@it=\textit\def\PY@tc##1{\textcolor[rgb]{0.25,0.50,0.50}{##1}}}

\def\PYZbs{\char`\\}
\def\PYZus{\char`\_}
\def\PYZob{\char`\{}
\def\PYZcb{\char`\}}
\def\PYZca{\char`\^}
\def\PYZam{\char`\&}
\def\PYZlt{\char`\<}
\def\PYZgt{\char`\>}
\def\PYZsh{\char`\#}
\def\PYZpc{\char`\%}
\def\PYZdl{\char`\$}
\def\PYZhy{\char`\-}
\def\PYZsq{\char`\'}
\def\PYZdq{\char`\"}
\def\PYZti{\char`\~}
% for compatibility with earlier versions
\def\PYZat{@}
\def\PYZlb{[}
\def\PYZrb{]}
\makeatother


    % For linebreaks inside Verbatim environment from package fancyvrb. 
    \makeatletter
        \newbox\Wrappedcontinuationbox 
        \newbox\Wrappedvisiblespacebox 
        \newcommand*\Wrappedvisiblespace {\textcolor{red}{\textvisiblespace}} 
        \newcommand*\Wrappedcontinuationsymbol {\textcolor{red}{\llap{\tiny$\m@th\hookrightarrow$}}} 
        \newcommand*\Wrappedcontinuationindent {3ex } 
        \newcommand*\Wrappedafterbreak {\kern\Wrappedcontinuationindent\copy\Wrappedcontinuationbox} 
        % Take advantage of the already applied Pygments mark-up to insert 
        % potential linebreaks for TeX processing. 
        %        {, <, #, %, $, ' and ": go to next line. 
        %        _, }, ^, &, >, - and ~: stay at end of broken line. 
        % Use of \textquotesingle for straight quote. 
        \newcommand*\Wrappedbreaksatspecials {% 
            \def\PYGZus{\discretionary{\char`\_}{\Wrappedafterbreak}{\char`\_}}% 
            \def\PYGZob{\discretionary{}{\Wrappedafterbreak\char`\{}{\char`\{}}% 
            \def\PYGZcb{\discretionary{\char`\}}{\Wrappedafterbreak}{\char`\}}}% 
            \def\PYGZca{\discretionary{\char`\^}{\Wrappedafterbreak}{\char`\^}}% 
            \def\PYGZam{\discretionary{\char`\&}{\Wrappedafterbreak}{\char`\&}}% 
            \def\PYGZlt{\discretionary{}{\Wrappedafterbreak\char`\<}{\char`\<}}% 
            \def\PYGZgt{\discretionary{\char`\>}{\Wrappedafterbreak}{\char`\>}}% 
            \def\PYGZsh{\discretionary{}{\Wrappedafterbreak\char`\#}{\char`\#}}% 
            \def\PYGZpc{\discretionary{}{\Wrappedafterbreak\char`\%}{\char`\%}}% 
            \def\PYGZdl{\discretionary{}{\Wrappedafterbreak\char`\$}{\char`\$}}% 
            \def\PYGZhy{\discretionary{\char`\-}{\Wrappedafterbreak}{\char`\-}}% 
            \def\PYGZsq{\discretionary{}{\Wrappedafterbreak\textquotesingle}{\textquotesingle}}% 
            \def\PYGZdq{\discretionary{}{\Wrappedafterbreak\char`\"}{\char`\"}}% 
            \def\PYGZti{\discretionary{\char`\~}{\Wrappedafterbreak}{\char`\~}}% 
        } 
        % Some characters . , ; ? ! / are not pygmentized. 
        % This macro makes them "active" and they will insert potential linebreaks 
        \newcommand*\Wrappedbreaksatpunct {% 
            \lccode`\~`\.\lowercase{\def~}{\discretionary{\hbox{\char`\.}}{\Wrappedafterbreak}{\hbox{\char`\.}}}% 
            \lccode`\~`\,\lowercase{\def~}{\discretionary{\hbox{\char`\,}}{\Wrappedafterbreak}{\hbox{\char`\,}}}% 
            \lccode`\~`\;\lowercase{\def~}{\discretionary{\hbox{\char`\;}}{\Wrappedafterbreak}{\hbox{\char`\;}}}% 
            \lccode`\~`\:\lowercase{\def~}{\discretionary{\hbox{\char`\:}}{\Wrappedafterbreak}{\hbox{\char`\:}}}% 
            \lccode`\~`\?\lowercase{\def~}{\discretionary{\hbox{\char`\?}}{\Wrappedafterbreak}{\hbox{\char`\?}}}% 
            \lccode`\~`\!\lowercase{\def~}{\discretionary{\hbox{\char`\!}}{\Wrappedafterbreak}{\hbox{\char`\!}}}% 
            \lccode`\~`\/\lowercase{\def~}{\discretionary{\hbox{\char`\/}}{\Wrappedafterbreak}{\hbox{\char`\/}}}% 
            \catcode`\.\active
            \catcode`\,\active 
            \catcode`\;\active
            \catcode`\:\active
            \catcode`\?\active
            \catcode`\!\active
            \catcode`\/\active 
            \lccode`\~`\~ 	
        }
    \makeatother

    \let\OriginalVerbatim=\Verbatim
    \makeatletter
    \renewcommand{\Verbatim}[1][1]{%
        %\parskip\z@skip
        \sbox\Wrappedcontinuationbox {\Wrappedcontinuationsymbol}%
        \sbox\Wrappedvisiblespacebox {\FV@SetupFont\Wrappedvisiblespace}%
        \def\FancyVerbFormatLine ##1{\hsize\linewidth
            \vtop{\raggedright\hyphenpenalty\z@\exhyphenpenalty\z@
                \doublehyphendemerits\z@\finalhyphendemerits\z@
                \strut ##1\strut}%
        }%
        % If the linebreak is at a space, the latter will be displayed as visible
        % space at end of first line, and a continuation symbol starts next line.
        % Stretch/shrink are however usually zero for typewriter font.
        \def\FV@Space {%
            \nobreak\hskip\z@ plus\fontdimen3\font minus\fontdimen4\font
            \discretionary{\copy\Wrappedvisiblespacebox}{\Wrappedafterbreak}
            {\kern\fontdimen2\font}%
        }%
        
        % Allow breaks at special characters using \PYG... macros.
        \Wrappedbreaksatspecials
        % Breaks at punctuation characters . , ; ? ! and / need catcode=\active 	
        \OriginalVerbatim[#1,codes*=\Wrappedbreaksatpunct]%
    }
    \makeatother

    % Exact colors from NB
    \definecolor{incolor}{HTML}{303F9F}
    \definecolor{outcolor}{HTML}{D84315}
    \definecolor{cellborder}{HTML}{CFCFCF}
    \definecolor{cellbackground}{HTML}{F7F7F7}
    
    % prompt
    \makeatletter
    \newcommand{\boxspacing}{\kern\kvtcb@left@rule\kern\kvtcb@boxsep}
    \makeatother
    \newcommand{\prompt}[4]{
        {\ttfamily\llap{{\color{#2}[#3]:\hspace{3pt}#4}}\vspace{-\baselineskip}}
    }
    

    
    % Prevent overflowing lines due to hard-to-break entities
    \sloppy 
    % Setup hyperref package
    \hypersetup{
      breaklinks=true,  % so long urls are correctly broken across lines
      colorlinks=true,
      urlcolor=urlcolor,
      linkcolor=linkcolor,
      citecolor=citecolor,
      }
    % Slightly bigger margins than the latex defaults
    
    \geometry{verbose,tmargin=1in,bmargin=1in,lmargin=1in,rmargin=1in}
    
    

\begin{document}
    
    \maketitle
    
    

    
    UFMG PROGRAMA DE PÓS-GRADUAÇÃO EM ENGENHARIA ELÉTRICA SISTEMAS NEBULOSOS
PROF. WALMIR MATOS CAMINHAS

\# Exercício Computacional: Parte 2

Aluno: Hélder Seixas Lima Belo Horizonte, agosto de 2021.

\hypertarget{informauxe7uxf5es-sobre-a-implementauxe7uxe3o}{%
\section{Informações sobre a
implementação}\label{informauxe7uxf5es-sobre-a-implementauxe7uxe3o}}

As estruturas ANFIS e NFN foram implementadas em Python 3 utilizando
Jupyter Notebook. Live Script também foram criados no MATLAB para
executar as estruturas GENFIS e ANFIS disponíveis nessa ferramenta. Os
códigos implementados estão anexados no final deste relatório e também
no Moodle juntamente com os dados utilizados.

    \begin{tcolorbox}[breakable, size=fbox, boxrule=1pt, pad at break*=1mm,colback=cellbackground, colframe=cellborder]
\prompt{In}{incolor}{1}{\boxspacing}
\begin{Verbatim}[commandchars=\\\{\}]
\PY{k+kn}{import} \PY{n+nn}{numpy} \PY{k}{as} \PY{n+nn}{np}
\PY{k+kn}{import} \PY{n+nn}{math}
\PY{k+kn}{import} \PY{n+nn}{pandas} \PY{k}{as} \PY{n+nn}{pd}
\PY{k+kn}{import} \PY{n+nn}{matplotlib}\PY{n+nn}{.}\PY{n+nn}{pyplot} \PY{k}{as} \PY{n+nn}{plt}
\PY{k+kn}{import} \PY{n+nn}{matplotlib}\PY{n+nn}{.}\PY{n+nn}{dates} \PY{k}{as} \PY{n+nn}{mdates}
\PY{k+kn}{import} \PY{n+nn}{scipy}\PY{n+nn}{.}\PY{n+nn}{io}
\PY{k+kn}{from} \PY{n+nn}{neuro\PYZus{}fuzzy}\PY{n+nn}{.}\PY{n+nn}{anfis} \PY{k+kn}{import} \PY{n}{ANFIS}
\PY{k+kn}{from} \PY{n+nn}{neuro\PYZus{}fuzzy}\PY{n+nn}{.}\PY{n+nn}{nfn} \PY{k+kn}{import} \PY{n}{NFN}
\PY{k+kn}{from} \PY{n+nn}{neuro\PYZus{}fuzzy}\PY{n+nn}{.}\PY{n+nn}{util} \PY{k+kn}{import} \PY{n}{normalizar}\PY{p}{,} \PY{n}{embaralhar\PYZus{}dados}\PY{p}{,} \PY{n}{exibir\PYZus{}resultado\PYZus{}desejado}\PY{p}{,} \PY{n}{atrasar}\PY{p}{,} \PY{n}{avancar}\PY{p}{,} \PY{n}{ajustar}\PY{p}{,} \PYZbs{}
    \PY{n}{criar\PYZus{}arquivo}\PY{p}{,} \PY{n}{desnormalizar}


\PY{k}{def} \PY{n+nf}{descrever\PYZus{}dados}\PY{p}{(}\PY{n}{xt}\PY{p}{,} \PY{n}{yt}\PY{p}{,} \PY{n}{xv}\PY{p}{,} \PY{n}{yv}\PY{p}{)}\PY{p}{:}
    \PY{n+nb}{print}\PY{p}{(}\PY{l+s+s1}{\PYZsq{}}\PY{l+s+se}{\PYZbs{}n}\PY{l+s+s1}{===Dataset xt===}\PY{l+s+s1}{\PYZsq{}}\PY{p}{)}\PY{p}{;}
    \PY{n+nb}{print}\PY{p}{(}\PY{n}{pd}\PY{o}{.}\PY{n}{DataFrame}\PY{p}{(}\PY{n}{xt}\PY{p}{,} \PY{n}{columns}\PY{o}{=}\PY{k+kc}{None}\PY{p}{)}\PY{p}{)}\PY{p}{;}
    \PY{n+nb}{print}\PY{p}{(}\PY{l+s+s1}{\PYZsq{}}\PY{l+s+se}{\PYZbs{}n}\PY{l+s+s1}{===Dataset yt===}\PY{l+s+s1}{\PYZsq{}}\PY{p}{)}\PY{p}{;}
    \PY{n+nb}{print}\PY{p}{(}\PY{n}{pd}\PY{o}{.}\PY{n}{DataFrame}\PY{p}{(}\PY{n}{yt}\PY{p}{,} \PY{n}{columns}\PY{o}{=}\PY{k+kc}{None}\PY{p}{)}\PY{p}{)}\PY{p}{;}
    \PY{n+nb}{print}\PY{p}{(}\PY{l+s+s1}{\PYZsq{}}\PY{l+s+se}{\PYZbs{}n}\PY{l+s+s1}{===Dataset xv===}\PY{l+s+s1}{\PYZsq{}}\PY{p}{)}\PY{p}{;}
    \PY{n+nb}{print}\PY{p}{(}\PY{n}{pd}\PY{o}{.}\PY{n}{DataFrame}\PY{p}{(}\PY{n}{xv}\PY{p}{,} \PY{n}{columns}\PY{o}{=}\PY{k+kc}{None}\PY{p}{)}\PY{p}{)}\PY{p}{;}
    \PY{n+nb}{print}\PY{p}{(}\PY{l+s+s1}{\PYZsq{}}\PY{l+s+se}{\PYZbs{}n}\PY{l+s+s1}{===Dataset yv===}\PY{l+s+s1}{\PYZsq{}}\PY{p}{)}\PY{p}{;}
    \PY{n+nb}{print}\PY{p}{(}\PY{n}{pd}\PY{o}{.}\PY{n}{DataFrame}\PY{p}{(}\PY{n}{yv}\PY{p}{,} \PY{n}{columns}\PY{o}{=}\PY{k+kc}{None}\PY{p}{)}\PY{p}{)}\PY{p}{;}
\end{Verbatim}
\end{tcolorbox}

    \hypertarget{anuxe1lise-da-complexidade-computacional-dos-sistemas-de-inferuxeancia-fuzzy}{%
\subsection{Análise da complexidade computacional dos Sistemas de
Inferência
Fuzzy}\label{anuxe1lise-da-complexidade-computacional-dos-sistemas-de-inferuxeancia-fuzzy}}

Nesta seção serão analisados os comportamentos assintóticos dos
algoritmos Adaptive Neuro-Fuzzy inference System (ANFIS) e
Neo-Fyzzy-Neuron (NFN) considerando como operação relevante a execução
de uma inferência nessas estruturas.

O ANFIS possui \(f(n, m) = \mathcal{O}(n * m)\), onde \(n\) é o número
de entradas do sistema e \(m\) é o número de regras do sistema.

Como no NFN o número de regras não interfere no tempo de exceução da
estrutura, dado que a cada inferência no máximo apenas 2 funções de
pertinências são computadas, temos que para o NFN a
\(f(n) = \mathcal{O}(n)\). Por ter tempo de execução menor, o NFN pode
ser interessante para contextos de aplciações em tempo real e em
treinamento on-line.

Cabe destacar que tanto no ANFIS e no NFN o processo de aprendizagem
também considera o número de épocas de treino (\(e\)) e o número de
pontos de treino (\(t\)). Assim, temos que a complexidade do processo de
aprendizagem do ANFIS e NFN é
\(f(e, t, n, m) = \mathcal{O}(e * t * n * m)\) e
\(f(e, t, n) = \mathcal{O}(e * t * n)\), respectivamente.

O MatLab disponibiliza os métodos Grid Partition, Subtractive Clustering
e FCM Clustering com intuito de otimizar a inicialização de funções de
pertinência e regras para estruturas ANFIS.

O método Grid Partition gera funções de pertinência particionando
uniformemente os intervalos das variáveis de entrada. É criada uma regra
para cada combinação de função de pertinência e entrada. Assim, o número
de regras corresponde a \(m = p^{n}\), onde \(p\) é o número de funções
de pertinência por entrada. Desta forma o custo de sistema ANFIS
inicializado com este método é \(f(n, p) = \mathcal{O}(p^{n})\),
tornando esta solução cara para problemas com muitas entradas.

O método Subtractive Clustering gera um sistema usando funções de
pertinência e regras derivadas de clusters de dados encontrados a partir
de algoritmo agrupamento subtrativo considerando dados de entrada e
saída. O número de funções de pertinências por entrada e número de
regras corresponde ao número de clusters encontrados pelo algoritmo.
Segundo Chiu (1994), o custo do algoritmo é
\(f(t) = \mathcal{O}(t^{2})\).

O método FCM Clustering gera um sistema usando funções de pertinência e
regras derivadas de clusters de dados encontrados usando algoritimo de
agrupamento FCM a partir de dados de entrada e saída. O número de
funções de pertinências por entrada e número de regras corresponde ao
número de clusters, sendo que o número de clusters é definido por
parâmetro. Segundo Ghosh e Dubey (2013), o custo do algoritmo é
\(f(t, n, c, i) = \mathcal{O}(t * n * c^{2} * i)\), onde \(c\) é o
número de clusters e \(i\) é o número de iterações do algoritmo.

    \hypertarget{problema-1-modelagem-de-sistema-estuxe1tico-monovariuxe1vel}{%
\subsection{Problema 1: (modelagem de sistema estático
monovariável)}\label{problema-1-modelagem-de-sistema-estuxe1tico-monovariuxe1vel}}

Aproximar a função \(y=x^2\).

    \hypertarget{carregar-dados-de-treino-e-validauxe7uxe3o}{%
\subsubsection{Carregar dados de treino e
validação}\label{carregar-dados-de-treino-e-validauxe7uxe3o}}

    \begin{tcolorbox}[breakable, size=fbox, boxrule=1pt, pad at break*=1mm,colback=cellbackground, colframe=cellborder]
\prompt{In}{incolor}{2}{\boxspacing}
\begin{Verbatim}[commandchars=\\\{\}]
\PY{n}{xt} \PY{o}{=} \PY{n}{scipy}\PY{o}{.}\PY{n}{io}\PY{o}{.}\PY{n}{loadmat}\PY{p}{(}\PY{l+s+s1}{\PYZsq{}}\PY{l+s+s1}{xte1.mat}\PY{l+s+s1}{\PYZsq{}}\PY{p}{)}\PY{p}{[}\PY{l+s+s1}{\PYZsq{}}\PY{l+s+s1}{xte1}\PY{l+s+s1}{\PYZsq{}}\PY{p}{]}
\PY{n}{ydt} \PY{o}{=} \PY{n}{scipy}\PY{o}{.}\PY{n}{io}\PY{o}{.}\PY{n}{loadmat}\PY{p}{(}\PY{l+s+s1}{\PYZsq{}}\PY{l+s+s1}{ydte1.mat}\PY{l+s+s1}{\PYZsq{}}\PY{p}{)}\PY{p}{[}\PY{l+s+s1}{\PYZsq{}}\PY{l+s+s1}{ydte1}\PY{l+s+s1}{\PYZsq{}}\PY{p}{]}\PY{o}{.}\PY{n}{T}\PY{p}{[}\PY{l+m+mi}{0}\PY{p}{]}
\PY{n}{xv} \PY{o}{=} \PY{n}{scipy}\PY{o}{.}\PY{n}{io}\PY{o}{.}\PY{n}{loadmat}\PY{p}{(}\PY{l+s+s1}{\PYZsq{}}\PY{l+s+s1}{xve1.mat}\PY{l+s+s1}{\PYZsq{}}\PY{p}{)}\PY{p}{[}\PY{l+s+s1}{\PYZsq{}}\PY{l+s+s1}{xve1}\PY{l+s+s1}{\PYZsq{}}\PY{p}{]}
\PY{n}{ydv} \PY{o}{=} \PY{n}{scipy}\PY{o}{.}\PY{n}{io}\PY{o}{.}\PY{n}{loadmat}\PY{p}{(}\PY{l+s+s1}{\PYZsq{}}\PY{l+s+s1}{ydve1.mat}\PY{l+s+s1}{\PYZsq{}}\PY{p}{)}\PY{p}{[}\PY{l+s+s1}{\PYZsq{}}\PY{l+s+s1}{ydve1}\PY{l+s+s1}{\PYZsq{}}\PY{p}{]}\PY{o}{.}\PY{n}{T}\PY{p}{[}\PY{l+m+mi}{0}\PY{p}{]}
\end{Verbatim}
\end{tcolorbox}

    \hypertarget{resultado-desejado}{%
\subsubsection{Resultado desejado}\label{resultado-desejado}}

    \begin{tcolorbox}[breakable, size=fbox, boxrule=1pt, pad at break*=1mm,colback=cellbackground, colframe=cellborder]
\prompt{In}{incolor}{3}{\boxspacing}
\begin{Verbatim}[commandchars=\\\{\}]
\PY{n}{label\PYZus{}y\PYZus{}validacao} \PY{o}{=} \PY{l+s+s2}{\PYZdq{}}\PY{l+s+s2}{x²}\PY{l+s+s2}{\PYZdq{}}
\PY{n}{exibir\PYZus{}resultado\PYZus{}desejado}\PY{p}{(}\PY{n}{xv}\PY{p}{,} \PY{n}{ydv}\PY{p}{,} \PY{n}{label\PYZus{}y\PYZus{}validacao}\PY{p}{)}
\end{Verbatim}
\end{tcolorbox}

    \begin{center}
    \adjustimage{max size={0.9\linewidth}{0.9\paperheight}}{ExercicioComputacionalParte2_files/ExercicioComputacionalParte2_7_0.png}
    \end{center}
    { \hspace*{\fill} \\}
    
    \hypertarget{resumo-dos-dados-de-treino-e-validauxe7uxe3o}{%
\subsubsection{Resumo dos dados de treino e
validação}\label{resumo-dos-dados-de-treino-e-validauxe7uxe3o}}

    \begin{tcolorbox}[breakable, size=fbox, boxrule=1pt, pad at break*=1mm,colback=cellbackground, colframe=cellborder]
\prompt{In}{incolor}{4}{\boxspacing}
\begin{Verbatim}[commandchars=\\\{\}]
\PY{n}{descrever\PYZus{}dados}\PY{p}{(}\PY{n}{xt}\PY{p}{,} \PY{n}{ydt}\PY{p}{,} \PY{n}{xv}\PY{p}{,} \PY{n}{ydv}\PY{p}{)}
\end{Verbatim}
\end{tcolorbox}

    \begin{Verbatim}[commandchars=\\\{\}]

===Dataset xt===
            0
0    0.151927
1   -1.305227
2   -1.964813
3   -0.095791
4   -1.701748
..        {\ldots}
495  0.899214
496  1.975956
497 -1.207648
498  0.010367
499 -0.151371

[500 rows x 1 columns]

===Dataset yt===
            0
0    0.023082
1    1.703617
2    3.860489
3    0.009176
4    2.895945
..        {\ldots}
495  0.808586
496  3.904403
497  1.458413
498  0.000107
499  0.022913

[500 rows x 1 columns]

===Dataset xv===
          0
0    -2.000
1    -1.999
2    -1.998
3    -1.997
4    -1.996
{\ldots}     {\ldots}
3996  1.996
3997  1.997
3998  1.998
3999  1.999
4000  2.000

[4001 rows x 1 columns]

===Dataset yv===
             0
0     4.000000
1     3.996001
2     3.992004
3     3.988009
4     3.984016
{\ldots}        {\ldots}
3996  3.984016
3997  3.988009
3998  3.992004
3999  3.996001
4000  4.000000

[4001 rows x 1 columns]
    \end{Verbatim}

    \hypertarget{estruturas-implementadas}{%
\subsubsection{Estruturas
implementadas}\label{estruturas-implementadas}}

\hypertarget{paruxe2metros-gerais}{%
\paragraph{Parâmetros gerais}\label{paruxe2metros-gerais}}

    \begin{tcolorbox}[breakable, size=fbox, boxrule=1pt, pad at break*=1mm,colback=cellbackground, colframe=cellborder]
\prompt{In}{incolor}{5}{\boxspacing}
\begin{Verbatim}[commandchars=\\\{\}]
\PY{n}{n\PYZus{}epoca} \PY{o}{=} \PY{l+m+mi}{20}
\end{Verbatim}
\end{tcolorbox}

    \hypertarget{estrutura-anfis}{%
\paragraph{Estrutura ANFIS}\label{estrutura-anfis}}

Parâmetros

    \begin{tcolorbox}[breakable, size=fbox, boxrule=1pt, pad at break*=1mm,colback=cellbackground, colframe=cellborder]
\prompt{In}{incolor}{6}{\boxspacing}
\begin{Verbatim}[commandchars=\\\{\}]
\PY{n}{m} \PY{o}{=} \PY{l+m+mi}{4}
\PY{n}{alfa} \PY{o}{=} \PY{l+m+mf}{0.3}
\PY{n}{anfis\PYZus{}problema\PYZus{}1} \PY{o}{=} \PY{n}{ANFIS}\PY{p}{(}\PY{n}{n\PYZus{}epoca}\PY{p}{,} \PY{n}{xt}\PY{p}{,} \PY{n}{ydt}\PY{p}{,} \PY{n}{xv}\PY{p}{,} \PY{n}{ydv}\PY{p}{,} \PY{n}{m}\PY{p}{,} \PY{n}{label\PYZus{}y\PYZus{}validacao}\PY{p}{,} \PY{n}{alfa}\PY{p}{)}
\end{Verbatim}
\end{tcolorbox}

    Demonstração de estado inicial

    \begin{tcolorbox}[breakable, size=fbox, boxrule=1pt, pad at break*=1mm,colback=cellbackground, colframe=cellborder]
\prompt{In}{incolor}{7}{\boxspacing}
\begin{Verbatim}[commandchars=\\\{\}]
\PY{n}{anfis\PYZus{}problema\PYZus{}1}\PY{o}{.}\PY{n}{exibir\PYZus{}resultado\PYZus{}validacao}\PY{p}{(}\PY{p}{)}
\end{Verbatim}
\end{tcolorbox}

    \begin{center}
    \adjustimage{max size={0.9\linewidth}{0.9\paperheight}}{ExercicioComputacionalParte2_files/ExercicioComputacionalParte2_15_0.png}
    \end{center}
    { \hspace*{\fill} \\}
    
    \begin{Verbatim}[commandchars=\\\{\}]
Root mean squared error (RMSE):  1.6713
Mean absolute percentage error (MAPE): 39212.5888\%
    \end{Verbatim}

    Demonstração de treinamento

    \begin{tcolorbox}[breakable, size=fbox, boxrule=1pt, pad at break*=1mm,colback=cellbackground, colframe=cellborder]
\prompt{In}{incolor}{8}{\boxspacing}
\begin{Verbatim}[commandchars=\\\{\}]
\PY{n}{anfis\PYZus{}problema\PYZus{}1}\PY{o}{.}\PY{n}{treinar\PYZus{}gradiente}\PY{p}{(}\PY{n}{plota\PYZus{}resultado\PYZus{}epocas}\PY{o}{=}\PY{k+kc}{True}\PY{p}{)}
\end{Verbatim}
\end{tcolorbox}

    \begin{center}
    \adjustimage{max size={0.9\linewidth}{0.9\paperheight}}{ExercicioComputacionalParte2_files/ExercicioComputacionalParte2_17_0.png}
    \end{center}
    { \hspace*{\fill} \\}
    
    \begin{center}
    \adjustimage{max size={0.9\linewidth}{0.9\paperheight}}{ExercicioComputacionalParte2_files/ExercicioComputacionalParte2_17_1.png}
    \end{center}
    { \hspace*{\fill} \\}
    
    Demonstração de resultado final

    \begin{tcolorbox}[breakable, size=fbox, boxrule=1pt, pad at break*=1mm,colback=cellbackground, colframe=cellborder]
\prompt{In}{incolor}{9}{\boxspacing}
\begin{Verbatim}[commandchars=\\\{\}]
\PY{n}{anfis\PYZus{}problema\PYZus{}1}\PY{o}{.}\PY{n}{exibir\PYZus{}resultado\PYZus{}validacao}\PY{p}{(}\PY{p}{)}
\end{Verbatim}
\end{tcolorbox}

    \begin{center}
    \adjustimage{max size={0.9\linewidth}{0.9\paperheight}}{ExercicioComputacionalParte2_files/ExercicioComputacionalParte2_19_0.png}
    \end{center}
    { \hspace*{\fill} \\}
    
    \begin{Verbatim}[commandchars=\\\{\}]
Root mean squared error (RMSE):  0.0049
Mean absolute percentage error (MAPE): 264.5898\%
    \end{Verbatim}

    Resultado médio para 10 treinos

    \begin{tcolorbox}[breakable, size=fbox, boxrule=1pt, pad at break*=1mm,colback=cellbackground, colframe=cellborder]
\prompt{In}{incolor}{10}{\boxspacing}
\begin{Verbatim}[commandchars=\\\{\}]
\PY{n}{anfis\PYZus{}problema\PYZus{}1}\PY{o}{.}\PY{n}{exibir\PYZus{}resultado\PYZus{}validacao\PYZus{}multiplos\PYZus{}treinos}\PY{p}{(}\PY{l+m+mi}{10}\PY{p}{)}
\end{Verbatim}
\end{tcolorbox}

    \begin{center}
    \adjustimage{max size={0.9\linewidth}{0.9\paperheight}}{ExercicioComputacionalParte2_files/ExercicioComputacionalParte2_21_0.png}
    \end{center}
    { \hspace*{\fill} \\}
    
    \begin{Verbatim}[commandchars=\\\{\}]
Média do RMSE:  0.0225
Desvio padrão do RMSE:  0.0109
    \end{Verbatim}

    \begin{center}
    \adjustimage{max size={0.9\linewidth}{0.9\paperheight}}{ExercicioComputacionalParte2_files/ExercicioComputacionalParte2_21_2.png}
    \end{center}
    { \hspace*{\fill} \\}
    
    \begin{Verbatim}[commandchars=\\\{\}]
Média do MAPE: 449.4414\%
Desvio padrão do MAPE: 251.8057\%
    \end{Verbatim}

    \hypertarget{estrutura-nfn}{%
\paragraph{Estrutura NFN}\label{estrutura-nfn}}

Parâmetros

    \begin{tcolorbox}[breakable, size=fbox, boxrule=1pt, pad at break*=1mm,colback=cellbackground, colframe=cellborder]
\prompt{In}{incolor}{11}{\boxspacing}
\begin{Verbatim}[commandchars=\\\{\}]
\PY{n}{m} \PY{o}{=} \PY{l+m+mi}{120}
\PY{n}{alfa} \PY{o}{=} \PY{l+m+mf}{0.8}
\PY{n}{nfn\PYZus{}problema\PYZus{}1} \PY{o}{=} \PY{n}{NFN}\PY{p}{(}\PY{n}{n\PYZus{}epoca}\PY{p}{,} \PY{n}{xt}\PY{p}{,} \PY{n}{ydt}\PY{p}{,} \PY{n}{xv}\PY{p}{,} \PY{n}{ydv}\PY{p}{,} \PY{n}{m}\PY{p}{,} \PY{n}{label\PYZus{}y\PYZus{}validacao}\PY{p}{,} \PY{n}{alfa}\PY{p}{)}
\end{Verbatim}
\end{tcolorbox}

    Demonstração de estado inicial

    \begin{tcolorbox}[breakable, size=fbox, boxrule=1pt, pad at break*=1mm,colback=cellbackground, colframe=cellborder]
\prompt{In}{incolor}{12}{\boxspacing}
\begin{Verbatim}[commandchars=\\\{\}]
\PY{n}{nfn\PYZus{}problema\PYZus{}1}\PY{o}{.}\PY{n}{exibir\PYZus{}resultado\PYZus{}validacao}\PY{p}{(}\PY{p}{)}
\end{Verbatim}
\end{tcolorbox}

    \begin{center}
    \adjustimage{max size={0.9\linewidth}{0.9\paperheight}}{ExercicioComputacionalParte2_files/ExercicioComputacionalParte2_25_0.png}
    \end{center}
    { \hspace*{\fill} \\}
    
    \begin{Verbatim}[commandchars=\\\{\}]
Root mean squared error (RMSE):  1.9177
Mean absolute percentage error (MAPE): 49810.0775\%
    \end{Verbatim}

    Demonstração de treinamento

    \begin{tcolorbox}[breakable, size=fbox, boxrule=1pt, pad at break*=1mm,colback=cellbackground, colframe=cellborder]
\prompt{In}{incolor}{13}{\boxspacing}
\begin{Verbatim}[commandchars=\\\{\}]
\PY{n}{nfn\PYZus{}problema\PYZus{}1}\PY{o}{.}\PY{n}{treinar\PYZus{}gradiente}\PY{p}{(}\PY{n}{plota\PYZus{}resultado\PYZus{}epocas}\PY{o}{=}\PY{k+kc}{True}\PY{p}{)}
\end{Verbatim}
\end{tcolorbox}

    \begin{center}
    \adjustimage{max size={0.9\linewidth}{0.9\paperheight}}{ExercicioComputacionalParte2_files/ExercicioComputacionalParte2_27_0.png}
    \end{center}
    { \hspace*{\fill} \\}
    
    \begin{center}
    \adjustimage{max size={0.9\linewidth}{0.9\paperheight}}{ExercicioComputacionalParte2_files/ExercicioComputacionalParte2_27_1.png}
    \end{center}
    { \hspace*{\fill} \\}
    
    Demonstração de resultado final

    \begin{tcolorbox}[breakable, size=fbox, boxrule=1pt, pad at break*=1mm,colback=cellbackground, colframe=cellborder]
\prompt{In}{incolor}{14}{\boxspacing}
\begin{Verbatim}[commandchars=\\\{\}]
\PY{n}{nfn\PYZus{}problema\PYZus{}1}\PY{o}{.}\PY{n}{exibir\PYZus{}resultado\PYZus{}validacao}\PY{p}{(}\PY{p}{)}
\end{Verbatim}
\end{tcolorbox}

    \begin{center}
    \adjustimage{max size={0.9\linewidth}{0.9\paperheight}}{ExercicioComputacionalParte2_files/ExercicioComputacionalParte2_29_0.png}
    \end{center}
    { \hspace*{\fill} \\}
    
    \begin{Verbatim}[commandchars=\\\{\}]
Root mean squared error (RMSE):  0.0004
Mean absolute percentage error (MAPE): 7.9826\%
    \end{Verbatim}

    Resultado médio para 10 treinos

    \begin{tcolorbox}[breakable, size=fbox, boxrule=1pt, pad at break*=1mm,colback=cellbackground, colframe=cellborder]
\prompt{In}{incolor}{15}{\boxspacing}
\begin{Verbatim}[commandchars=\\\{\}]
\PY{n}{nfn\PYZus{}problema\PYZus{}1}\PY{o}{.}\PY{n}{exibir\PYZus{}resultado\PYZus{}validacao\PYZus{}multiplos\PYZus{}treinos}\PY{p}{(}\PY{l+m+mi}{10}\PY{p}{)}
\end{Verbatim}
\end{tcolorbox}

    \begin{center}
    \adjustimage{max size={0.9\linewidth}{0.9\paperheight}}{ExercicioComputacionalParte2_files/ExercicioComputacionalParte2_31_0.png}
    \end{center}
    { \hspace*{\fill} \\}
    
    \begin{Verbatim}[commandchars=\\\{\}]
Média do RMSE:  0.0004
Desvio padrão do RMSE:  0.0
    \end{Verbatim}

    \begin{center}
    \adjustimage{max size={0.9\linewidth}{0.9\paperheight}}{ExercicioComputacionalParte2_files/ExercicioComputacionalParte2_31_2.png}
    \end{center}
    { \hspace*{\fill} \\}
    
    \begin{Verbatim}[commandchars=\\\{\}]
Média do MAPE: 7.9822\%
Desvio padrão do MAPE: 0.0019\%
    \end{Verbatim}

    \hypertarget{problema-2-exemplo-2-do-livro-texto-modelagem-de-sistema-estuxe1tico-multivariuxe1vel}{%
\subsection{Problema 2: Exemplo 2 do livro texto (modelagem de sistema
estático
multivariável)}\label{problema-2-exemplo-2-do-livro-texto-modelagem-de-sistema-estuxe1tico-multivariuxe1vel}}

Aproximar a função
\(f(x, y, z) = 1 + x_{1}^{0,5} + x_{2}^{-1} + x_{3}^{1,5}\).

    \hypertarget{carregamento-dos-dados-de-treino-e-validauxe7uxe3o}{%
\subsubsection{Carregamento dos dados de treino e
validação}\label{carregamento-dos-dados-de-treino-e-validauxe7uxe3o}}

    \begin{tcolorbox}[breakable, size=fbox, boxrule=1pt, pad at break*=1mm,colback=cellbackground, colframe=cellborder]
\prompt{In}{incolor}{16}{\boxspacing}
\begin{Verbatim}[commandchars=\\\{\}]
\PY{n}{xt} \PY{o}{=} \PY{n}{scipy}\PY{o}{.}\PY{n}{io}\PY{o}{.}\PY{n}{loadmat}\PY{p}{(}\PY{l+s+s1}{\PYZsq{}}\PY{l+s+s1}{xte2.mat}\PY{l+s+s1}{\PYZsq{}}\PY{p}{)}\PY{p}{[}\PY{l+s+s1}{\PYZsq{}}\PY{l+s+s1}{xte2}\PY{l+s+s1}{\PYZsq{}}\PY{p}{]}\PY{o}{.}\PY{n}{tolist}\PY{p}{(}\PY{p}{)}
\PY{n}{ydt} \PY{o}{=} \PY{n}{scipy}\PY{o}{.}\PY{n}{io}\PY{o}{.}\PY{n}{loadmat}\PY{p}{(}\PY{l+s+s1}{\PYZsq{}}\PY{l+s+s1}{ydte2.mat}\PY{l+s+s1}{\PYZsq{}}\PY{p}{)}\PY{p}{[}\PY{l+s+s1}{\PYZsq{}}\PY{l+s+s1}{ydte2}\PY{l+s+s1}{\PYZsq{}}\PY{p}{]}\PY{o}{.}\PY{n}{T}\PY{p}{[}\PY{l+m+mi}{0}\PY{p}{]}\PY{o}{.}\PY{n}{tolist}\PY{p}{(}\PY{p}{)}
\PY{n}{xv} \PY{o}{=} \PY{n}{scipy}\PY{o}{.}\PY{n}{io}\PY{o}{.}\PY{n}{loadmat}\PY{p}{(}\PY{l+s+s1}{\PYZsq{}}\PY{l+s+s1}{xve2.mat}\PY{l+s+s1}{\PYZsq{}}\PY{p}{)}\PY{p}{[}\PY{l+s+s1}{\PYZsq{}}\PY{l+s+s1}{xve2}\PY{l+s+s1}{\PYZsq{}}\PY{p}{]}\PY{o}{.}\PY{n}{tolist}\PY{p}{(}\PY{p}{)}
\PY{n}{ydv} \PY{o}{=} \PY{n}{scipy}\PY{o}{.}\PY{n}{io}\PY{o}{.}\PY{n}{loadmat}\PY{p}{(}\PY{l+s+s1}{\PYZsq{}}\PY{l+s+s1}{ydve2.mat}\PY{l+s+s1}{\PYZsq{}}\PY{p}{)}\PY{p}{[}\PY{l+s+s1}{\PYZsq{}}\PY{l+s+s1}{ydve2}\PY{l+s+s1}{\PYZsq{}}\PY{p}{]}\PY{o}{.}\PY{n}{T}\PY{p}{[}\PY{l+m+mi}{0}\PY{p}{]}\PY{o}{.}\PY{n}{tolist}\PY{p}{(}\PY{p}{)}
\end{Verbatim}
\end{tcolorbox}

    \hypertarget{resultado-desejado}{%
\subsubsection{Resultado desejado}\label{resultado-desejado}}

    \begin{tcolorbox}[breakable, size=fbox, boxrule=1pt, pad at break*=1mm,colback=cellbackground, colframe=cellborder]
\prompt{In}{incolor}{17}{\boxspacing}
\begin{Verbatim}[commandchars=\\\{\}]
\PY{n}{label\PYZus{}y\PYZus{}validacao} \PY{o}{=} \PY{l+s+s2}{\PYZdq{}}\PY{l+s+s2}{\PYZdl{}1 + x\PYZus{}}\PY{l+s+si}{\PYZob{}1\PYZcb{}}\PY{l+s+s2}{\PYZca{}}\PY{l+s+s2}{\PYZob{}}\PY{l+s+s2}{0,5\PYZcb{} + x\PYZus{}}\PY{l+s+si}{\PYZob{}2\PYZcb{}}\PY{l+s+s2}{\PYZca{}}\PY{l+s+s2}{\PYZob{}}\PY{l+s+s2}{\PYZhy{}1\PYZcb{} + x\PYZus{}}\PY{l+s+si}{\PYZob{}3\PYZcb{}}\PY{l+s+s2}{\PYZca{}}\PY{l+s+s2}{\PYZob{}}\PY{l+s+s2}{1,5\PYZcb{}\PYZdl{}}\PY{l+s+s2}{\PYZdq{}}
\PY{n}{exibir\PYZus{}resultado\PYZus{}desejado}\PY{p}{(}\PY{n}{xv}\PY{p}{,} \PY{n}{ydv}\PY{p}{,} \PY{n}{label\PYZus{}y\PYZus{}validacao}\PY{p}{)}
\end{Verbatim}
\end{tcolorbox}

    \begin{center}
    \adjustimage{max size={0.9\linewidth}{0.9\paperheight}}{ExercicioComputacionalParte2_files/ExercicioComputacionalParte2_36_0.png}
    \end{center}
    { \hspace*{\fill} \\}
    
    \hypertarget{resumo-dos-dados-de-treino-e-validauxe7uxe3o}{%
\subsubsection{Resumo dos dados de treino e
validação}\label{resumo-dos-dados-de-treino-e-validauxe7uxe3o}}

    \begin{tcolorbox}[breakable, size=fbox, boxrule=1pt, pad at break*=1mm,colback=cellbackground, colframe=cellborder]
\prompt{In}{incolor}{18}{\boxspacing}
\begin{Verbatim}[commandchars=\\\{\}]
\PY{n}{descrever\PYZus{}dados}\PY{p}{(}\PY{n}{xt}\PY{p}{,} \PY{n}{ydt}\PY{p}{,} \PY{n}{xv}\PY{p}{,} \PY{n}{ydv}\PY{p}{)}
\end{Verbatim}
\end{tcolorbox}

    \begin{Verbatim}[commandchars=\\\{\}]

===Dataset xt===
     0  1  2
0    4  3  4
1    3  5  2
2    2  3  3
3    2  2  2
4    6  6  2
..  .. .. ..
211  5  4  4
212  2  1  1
213  2  5  3
214  3  6  3
215  2  5  6

[216 rows x 3 columns]

===Dataset yt===
             0
0    11.960069
1    10.795195
2     8.643582
3    10.678301
4    15.758596
..         {\ldots}
211  13.039812
212  19.485281
213   7.877361
214   9.555317
215   7.194492

[216 rows x 1 columns]

===Dataset xv===
       0    1    2
0    1.5  1.5  1.5
1    1.5  1.5  2.5
2    1.5  1.5  3.5
3    1.5  1.5  4.5
4    1.5  1.5  5.5
..   {\ldots}  {\ldots}  {\ldots}
120  5.5  5.5  1.5
121  5.5  5.5  2.5
122  5.5  5.5  3.5
123  5.5  5.5  4.5
124  5.5  5.5  5.5

[125 rows x 3 columns]

===Dataset yv===
             0
0    11.804327
1     9.887212
2     9.266741
3     8.977023
4     8.814599
..         {\ldots}
120  16.575949
121  14.288463
122  13.540536
123  13.189845
124  12.992807

[125 rows x 1 columns]
    \end{Verbatim}

    \hypertarget{estruturas-implementadas}{%
\subsubsection{Estruturas
implementadas}\label{estruturas-implementadas}}

\hypertarget{paruxe2metros-gerais}{%
\paragraph{Parâmetros gerais}\label{paruxe2metros-gerais}}

    \begin{tcolorbox}[breakable, size=fbox, boxrule=1pt, pad at break*=1mm,colback=cellbackground, colframe=cellborder]
\prompt{In}{incolor}{19}{\boxspacing}
\begin{Verbatim}[commandchars=\\\{\}]
\PY{n}{n\PYZus{}epoca} \PY{o}{=} \PY{l+m+mi}{20}
\end{Verbatim}
\end{tcolorbox}

    \hypertarget{estrutura-anfis}{%
\paragraph{Estrutura ANFIS}\label{estrutura-anfis}}

Parâmetros

    \begin{tcolorbox}[breakable, size=fbox, boxrule=1pt, pad at break*=1mm,colback=cellbackground, colframe=cellborder]
\prompt{In}{incolor}{20}{\boxspacing}
\begin{Verbatim}[commandchars=\\\{\}]
\PY{n}{m} \PY{o}{=} \PY{l+m+mi}{5}
\PY{n}{alfa} \PY{o}{=} \PY{l+m+mf}{0.01}
\PY{n}{anfis\PYZus{}problema\PYZus{}2} \PY{o}{=} \PY{n}{ANFIS}\PY{p}{(}\PY{n}{n\PYZus{}epoca}\PY{p}{,} \PY{n}{xt}\PY{p}{,} \PY{n}{ydt}\PY{p}{,} \PY{n}{xv}\PY{p}{,} \PY{n}{ydv}\PY{p}{,} \PY{n}{m}\PY{p}{,} \PY{n}{label\PYZus{}y\PYZus{}validacao}\PY{p}{,} \PY{n}{alfa}\PY{p}{)}
\end{Verbatim}
\end{tcolorbox}

    Demonstração de estado inicial

    \begin{tcolorbox}[breakable, size=fbox, boxrule=1pt, pad at break*=1mm,colback=cellbackground, colframe=cellborder]
\prompt{In}{incolor}{21}{\boxspacing}
\begin{Verbatim}[commandchars=\\\{\}]
\PY{n}{anfis\PYZus{}problema\PYZus{}2}\PY{o}{.}\PY{n}{exibir\PYZus{}resultado\PYZus{}validacao}\PY{p}{(}\PY{p}{)}
\end{Verbatim}
\end{tcolorbox}

    \begin{center}
    \adjustimage{max size={0.9\linewidth}{0.9\paperheight}}{ExercicioComputacionalParte2_files/ExercicioComputacionalParte2_44_0.png}
    \end{center}
    { \hspace*{\fill} \\}
    
    \begin{Verbatim}[commandchars=\\\{\}]
Root mean squared error (RMSE):  7.1365
Mean absolute percentage error (MAPE): 50.0936\%
    \end{Verbatim}

    Demonstração de treinamento

    \begin{tcolorbox}[breakable, size=fbox, boxrule=1pt, pad at break*=1mm,colback=cellbackground, colframe=cellborder]
\prompt{In}{incolor}{22}{\boxspacing}
\begin{Verbatim}[commandchars=\\\{\}]
\PY{n}{anfis\PYZus{}problema\PYZus{}2}\PY{o}{.}\PY{n}{treinar\PYZus{}gradiente}\PY{p}{(}\PY{n}{plota\PYZus{}resultado\PYZus{}epocas}\PY{o}{=}\PY{k+kc}{True}\PY{p}{)}
\end{Verbatim}
\end{tcolorbox}

    \begin{center}
    \adjustimage{max size={0.9\linewidth}{0.9\paperheight}}{ExercicioComputacionalParte2_files/ExercicioComputacionalParte2_46_0.png}
    \end{center}
    { \hspace*{\fill} \\}
    
    \begin{center}
    \adjustimage{max size={0.9\linewidth}{0.9\paperheight}}{ExercicioComputacionalParte2_files/ExercicioComputacionalParte2_46_1.png}
    \end{center}
    { \hspace*{\fill} \\}
    
    Demonstração de resultado final

    \begin{tcolorbox}[breakable, size=fbox, boxrule=1pt, pad at break*=1mm,colback=cellbackground, colframe=cellborder]
\prompt{In}{incolor}{23}{\boxspacing}
\begin{Verbatim}[commandchars=\\\{\}]
\PY{n}{anfis\PYZus{}problema\PYZus{}2}\PY{o}{.}\PY{n}{exibir\PYZus{}resultado\PYZus{}validacao}\PY{p}{(}\PY{p}{)}
\end{Verbatim}
\end{tcolorbox}

    \begin{center}
    \adjustimage{max size={0.9\linewidth}{0.9\paperheight}}{ExercicioComputacionalParte2_files/ExercicioComputacionalParte2_48_0.png}
    \end{center}
    { \hspace*{\fill} \\}
    
    \begin{Verbatim}[commandchars=\\\{\}]
Root mean squared error (RMSE):  1.1496
Mean absolute percentage error (MAPE): 5.5745\%
    \end{Verbatim}

    Resultado médio para 10 treinos

    \begin{tcolorbox}[breakable, size=fbox, boxrule=1pt, pad at break*=1mm,colback=cellbackground, colframe=cellborder]
\prompt{In}{incolor}{24}{\boxspacing}
\begin{Verbatim}[commandchars=\\\{\}]
\PY{n}{anfis\PYZus{}problema\PYZus{}2}\PY{o}{.}\PY{n}{exibir\PYZus{}resultado\PYZus{}validacao\PYZus{}multiplos\PYZus{}treinos}\PY{p}{(}\PY{l+m+mi}{10}\PY{p}{)}
\end{Verbatim}
\end{tcolorbox}

    \begin{center}
    \adjustimage{max size={0.9\linewidth}{0.9\paperheight}}{ExercicioComputacionalParte2_files/ExercicioComputacionalParte2_50_0.png}
    \end{center}
    { \hspace*{\fill} \\}
    
    \begin{Verbatim}[commandchars=\\\{\}]
Média do RMSE:  1.0061
Desvio padrão do RMSE:  0.1315
    \end{Verbatim}

    \begin{center}
    \adjustimage{max size={0.9\linewidth}{0.9\paperheight}}{ExercicioComputacionalParte2_files/ExercicioComputacionalParte2_50_2.png}
    \end{center}
    { \hspace*{\fill} \\}
    
    \begin{Verbatim}[commandchars=\\\{\}]
Média do MAPE: 5.5908\%
Desvio padrão do MAPE: 0.6504\%
    \end{Verbatim}

    \hypertarget{estrutura-nfn}{%
\paragraph{Estrutura NFN}\label{estrutura-nfn}}

Parâmetros

    \begin{tcolorbox}[breakable, size=fbox, boxrule=1pt, pad at break*=1mm,colback=cellbackground, colframe=cellborder]
\prompt{In}{incolor}{25}{\boxspacing}
\begin{Verbatim}[commandchars=\\\{\}]
\PY{n}{m} \PY{o}{=} \PY{l+m+mi}{5}
\PY{n}{alfa} \PY{o}{=} \PY{l+m+mf}{0.01}
\PY{n}{nfn\PYZus{}problema\PYZus{}2} \PY{o}{=} \PY{n}{NFN}\PY{p}{(}\PY{n}{n\PYZus{}epoca}\PY{p}{,} \PY{n}{xt}\PY{p}{,} \PY{n}{ydt}\PY{p}{,} \PY{n}{xv}\PY{p}{,} \PY{n}{ydv}\PY{p}{,} \PY{n}{m}\PY{p}{,} \PY{n}{label\PYZus{}y\PYZus{}validacao}\PY{p}{,} \PY{n}{alfa}\PY{p}{)}
\end{Verbatim}
\end{tcolorbox}

    Demonstração de estado inicial

    \begin{tcolorbox}[breakable, size=fbox, boxrule=1pt, pad at break*=1mm,colback=cellbackground, colframe=cellborder]
\prompt{In}{incolor}{26}{\boxspacing}
\begin{Verbatim}[commandchars=\\\{\}]
\PY{n}{nfn\PYZus{}problema\PYZus{}2}\PY{o}{.}\PY{n}{exibir\PYZus{}resultado\PYZus{}validacao}\PY{p}{(}\PY{p}{)}
\end{Verbatim}
\end{tcolorbox}

    \begin{center}
    \adjustimage{max size={0.9\linewidth}{0.9\paperheight}}{ExercicioComputacionalParte2_files/ExercicioComputacionalParte2_54_0.png}
    \end{center}
    { \hspace*{\fill} \\}
    
    \begin{Verbatim}[commandchars=\\\{\}]
Root mean squared error (RMSE):  4.1896
Mean absolute percentage error (MAPE): 28.6446\%
    \end{Verbatim}

    Demonstração de treinamento

    \begin{tcolorbox}[breakable, size=fbox, boxrule=1pt, pad at break*=1mm,colback=cellbackground, colframe=cellborder]
\prompt{In}{incolor}{27}{\boxspacing}
\begin{Verbatim}[commandchars=\\\{\}]
\PY{n}{nfn\PYZus{}problema\PYZus{}2}\PY{o}{.}\PY{n}{treinar\PYZus{}gradiente}\PY{p}{(}\PY{n}{plota\PYZus{}resultado\PYZus{}epocas}\PY{o}{=}\PY{k+kc}{True}\PY{p}{)}
\end{Verbatim}
\end{tcolorbox}

    \begin{center}
    \adjustimage{max size={0.9\linewidth}{0.9\paperheight}}{ExercicioComputacionalParte2_files/ExercicioComputacionalParte2_56_0.png}
    \end{center}
    { \hspace*{\fill} \\}
    
    \begin{center}
    \adjustimage{max size={0.9\linewidth}{0.9\paperheight}}{ExercicioComputacionalParte2_files/ExercicioComputacionalParte2_56_1.png}
    \end{center}
    { \hspace*{\fill} \\}
    
    Demonstração de resultado final

    \begin{tcolorbox}[breakable, size=fbox, boxrule=1pt, pad at break*=1mm,colback=cellbackground, colframe=cellborder]
\prompt{In}{incolor}{28}{\boxspacing}
\begin{Verbatim}[commandchars=\\\{\}]
\PY{n}{nfn\PYZus{}problema\PYZus{}2}\PY{o}{.}\PY{n}{exibir\PYZus{}resultado\PYZus{}validacao}\PY{p}{(}\PY{p}{)}
\end{Verbatim}
\end{tcolorbox}

    \begin{center}
    \adjustimage{max size={0.9\linewidth}{0.9\paperheight}}{ExercicioComputacionalParte2_files/ExercicioComputacionalParte2_58_0.png}
    \end{center}
    { \hspace*{\fill} \\}
    
    \begin{Verbatim}[commandchars=\\\{\}]
Root mean squared error (RMSE):  0.7369
Mean absolute percentage error (MAPE): 5.1652\%
    \end{Verbatim}

    Resultado médio para 10 treinos

    \begin{tcolorbox}[breakable, size=fbox, boxrule=1pt, pad at break*=1mm,colback=cellbackground, colframe=cellborder]
\prompt{In}{incolor}{29}{\boxspacing}
\begin{Verbatim}[commandchars=\\\{\}]
\PY{n}{nfn\PYZus{}problema\PYZus{}2}\PY{o}{.}\PY{n}{exibir\PYZus{}resultado\PYZus{}validacao\PYZus{}multiplos\PYZus{}treinos}\PY{p}{(}\PY{l+m+mi}{10}\PY{p}{)}
\end{Verbatim}
\end{tcolorbox}

    \begin{center}
    \adjustimage{max size={0.9\linewidth}{0.9\paperheight}}{ExercicioComputacionalParte2_files/ExercicioComputacionalParte2_60_0.png}
    \end{center}
    { \hspace*{\fill} \\}
    
    \begin{Verbatim}[commandchars=\\\{\}]
Média do RMSE:  0.7341
Desvio padrão do RMSE:  0.0018
    \end{Verbatim}

    \begin{center}
    \adjustimage{max size={0.9\linewidth}{0.9\paperheight}}{ExercicioComputacionalParte2_files/ExercicioComputacionalParte2_60_2.png}
    \end{center}
    { \hspace*{\fill} \\}
    
    \begin{Verbatim}[commandchars=\\\{\}]
Média do MAPE: 5.1261\%
Desvio padrão do MAPE: 0.0310\%
    \end{Verbatim}

    \hypertarget{problema-3-modelo-de-sistema-dinuxe2mico}{%
\subsection{Problema 3: Modelo de sistema
dinâmico}\label{problema-3-modelo-de-sistema-dinuxe2mico}}

Considere o sistema dinâmico descrito por:
\(y(k+1) = g[y(k), y(k-1), y(k-2), u(k), u(k-1)]\), onde
\(g(x_{1}, x_{2}, x_{3}, x_{4}, x_{5}) = \frac{x_{1}x_{2}x_{3}x_{5}(x_{3}-1)+x_{4}}{1+x_{3}^2+x_{4}^2}\)
e
\(u(k) = \{\begin{array}{ll}sen(2\Pi k/250), \forall k \leq 500 \\0.8sen(2\Pi k/250) + 0.2sen(2\Pi k/25), \forall k > 500 \end{array}\)

\(\therefore \hat{y}(k+1) = \hat{g}[\hat{y}(k), \hat{y}(k-1), \hat{y}(k-2), u(k), u(k-1)]\)

\begin{itemize}
\tightlist
\item
  Estrutura da rede neurofuzzy

  \begin{itemize}
  \tightlist
  \item
    cinco entradas \(y(k), y(k-1), y(k-2), u(k), u(k-1)\)
  \item
    uma saída \(y(k+1) = \hat{g}[\cdot ]\)
  \end{itemize}
\item
  Metodologia de treinamento sugerida

  \begin{itemize}
  \tightlist
  \item
    Utilizar cinco mil padrões com \(u(k)\) gerado aleatoriamente no
    intervalo \([-1,1]\).
  \end{itemize}
\end{itemize}

    \hypertarget{criauxe7uxe3o-dos-arquivos-de-treino-e-validauxe7uxe3o}{%
\subsubsection{Criação dos arquivos de treino e
validação}\label{criauxe7uxe3o-dos-arquivos-de-treino-e-validauxe7uxe3o}}

    \begin{tcolorbox}[breakable, size=fbox, boxrule=1pt, pad at break*=1mm,colback=cellbackground, colframe=cellborder]
\prompt{In}{incolor}{30}{\boxspacing}
\begin{Verbatim}[commandchars=\\\{\}]
\PY{k}{def} \PY{n+nf}{problema\PYZus{}3\PYZus{}calcular\PYZus{}g}\PY{p}{(}\PY{n}{x1}\PY{p}{,} \PY{n}{x2}\PY{p}{,} \PY{n}{x3}\PY{p}{,} \PY{n}{x4}\PY{p}{,} \PY{n}{x5}\PY{p}{)}\PY{p}{:}
    \PY{n}{g} \PY{o}{=} \PY{p}{(}\PY{n}{x1} \PY{o}{*} \PY{n}{x2} \PY{o}{*} \PY{n}{x3} \PY{o}{*} \PY{n}{x5} \PY{o}{*} \PY{p}{(}\PY{n}{x3} \PY{o}{\PYZhy{}} \PY{l+m+mi}{1}\PY{p}{)} \PY{o}{+} \PY{n}{x4}\PY{p}{)} \PY{o}{/} \PY{p}{(}\PY{l+m+mi}{1} \PY{o}{+} \PY{n}{x3} \PY{o}{*}\PY{o}{*} \PY{l+m+mi}{2} \PY{o}{+} \PY{n}{x4} \PY{o}{*}\PY{o}{*} \PY{l+m+mi}{2}\PY{p}{)}
    \PY{k}{return} \PY{n}{g}


\PY{k}{def} \PY{n+nf}{problema\PYZus{}3\PYZus{}calcular\PYZus{}u}\PY{p}{(}\PY{n}{k}\PY{p}{)}\PY{p}{:}
    \PY{k}{if} \PY{n}{k} \PY{o}{\PYZlt{}} \PY{l+m+mi}{0}\PY{p}{:}
        \PY{k}{return} \PY{l+m+mi}{0}
    \PY{k}{elif} \PY{n}{k} \PY{o}{\PYZlt{}}\PY{o}{=} \PY{l+m+mi}{500}\PY{p}{:}
        \PY{k}{return} \PY{n}{np}\PY{o}{.}\PY{n}{sin}\PY{p}{(}\PY{p}{(}\PY{l+m+mi}{2} \PY{o}{*} \PY{n}{math}\PY{o}{.}\PY{n}{pi} \PY{o}{*} \PY{n}{k}\PY{p}{)} \PY{o}{/} \PY{l+m+mi}{250}\PY{p}{)}
    \PY{k}{else}\PY{p}{:}
        \PY{k}{return} \PY{l+m+mf}{0.8} \PY{o}{*} \PY{n}{np}\PY{o}{.}\PY{n}{sin}\PY{p}{(}\PY{p}{(}\PY{l+m+mi}{2} \PY{o}{*} \PY{n}{math}\PY{o}{.}\PY{n}{pi} \PY{o}{*} \PY{n}{k}\PY{p}{)} \PY{o}{/} \PY{l+m+mi}{250}\PY{p}{)} \PY{o}{+} \PY{l+m+mf}{0.2} \PY{o}{*} \PY{n}{np}\PY{o}{.}\PY{n}{sin}\PY{p}{(}\PY{p}{(}\PY{l+m+mi}{2} \PY{o}{*} \PY{n}{math}\PY{o}{.}\PY{n}{pi} \PY{o}{*} \PY{n}{k}\PY{p}{)} \PY{o}{/} \PY{l+m+mi}{25}\PY{p}{)}


\PY{k}{def} \PY{n+nf}{problema\PYZus{}3\PYZus{}calcular\PYZus{}y}\PY{p}{(}\PY{n}{k}\PY{p}{,} \PY{n}{x}\PY{p}{,} \PY{n}{yd}\PY{p}{)}\PY{p}{:}
    \PY{k}{if} \PY{n}{k} \PY{o}{\PYZlt{}} \PY{l+m+mi}{0}\PY{p}{:}
        \PY{k}{return} \PY{l+m+mi}{0}\PY{p}{,} \PY{l+m+mi}{0}\PY{p}{,} \PY{l+m+mi}{0}\PY{p}{,} \PY{l+m+mi}{0}\PY{p}{,} \PY{l+m+mi}{0}\PY{p}{,} \PY{l+m+mi}{0}
    \PY{k}{elif} \PY{n}{k} \PY{o}{\PYZlt{}} \PY{n+nb}{len}\PY{p}{(}\PY{n}{yd}\PY{p}{)}\PY{p}{:}
        \PY{k}{return} \PY{n}{yd}\PY{p}{[}\PY{n}{k}\PY{p}{]}\PY{p}{[}\PY{l+m+mi}{0}\PY{p}{]}\PY{p}{,} \PY{n}{x}\PY{p}{[}\PY{n}{k}\PY{p}{]}\PY{p}{[}\PY{l+m+mi}{0}\PY{p}{]}\PY{p}{,} \PY{n}{x}\PY{p}{[}\PY{n}{k}\PY{p}{]}\PY{p}{[}\PY{l+m+mi}{1}\PY{p}{]}\PY{p}{,} \PY{n}{x}\PY{p}{[}\PY{n}{k}\PY{p}{]}\PY{p}{[}\PY{l+m+mi}{2}\PY{p}{]}\PY{p}{,} \PY{n}{x}\PY{p}{[}\PY{n}{k}\PY{p}{]}\PY{p}{[}\PY{l+m+mi}{3}\PY{p}{]}\PY{p}{,} \PY{n}{x}\PY{p}{[}\PY{n}{k}\PY{p}{]}\PY{p}{[}\PY{l+m+mi}{4}\PY{p}{]}
    \PY{k}{else}\PY{p}{:}
        \PY{n}{x1}\PY{p}{,} \PY{n}{\PYZus{}}\PY{p}{,} \PY{n}{\PYZus{}}\PY{p}{,} \PY{n}{\PYZus{}}\PY{p}{,} \PY{n}{\PYZus{}}\PY{p}{,} \PY{n}{\PYZus{}} \PY{o}{=} \PY{n}{problema\PYZus{}3\PYZus{}calcular\PYZus{}y}\PY{p}{(}\PY{n}{k} \PY{o}{\PYZhy{}} \PY{l+m+mi}{1}\PY{p}{,} \PY{n}{x}\PY{p}{,} \PY{n}{yd}\PY{p}{)}
        \PY{n}{x2}\PY{p}{,} \PY{n}{\PYZus{}}\PY{p}{,} \PY{n}{\PYZus{}}\PY{p}{,} \PY{n}{\PYZus{}}\PY{p}{,} \PY{n}{\PYZus{}}\PY{p}{,} \PY{n}{\PYZus{}} \PY{o}{=} \PY{n}{problema\PYZus{}3\PYZus{}calcular\PYZus{}y}\PY{p}{(}\PY{n}{k} \PY{o}{\PYZhy{}} \PY{l+m+mi}{2}\PY{p}{,} \PY{n}{x}\PY{p}{,} \PY{n}{yd}\PY{p}{)}
        \PY{n}{x3}\PY{p}{,} \PY{n}{\PYZus{}}\PY{p}{,} \PY{n}{\PYZus{}}\PY{p}{,} \PY{n}{\PYZus{}}\PY{p}{,} \PY{n}{\PYZus{}}\PY{p}{,} \PY{n}{\PYZus{}} \PY{o}{=} \PY{n}{problema\PYZus{}3\PYZus{}calcular\PYZus{}y}\PY{p}{(}\PY{n}{k} \PY{o}{\PYZhy{}} \PY{l+m+mi}{3}\PY{p}{,} \PY{n}{x}\PY{p}{,} \PY{n}{yd}\PY{p}{)}
        \PY{n}{x4} \PY{o}{=} \PY{n}{problema\PYZus{}3\PYZus{}calcular\PYZus{}u}\PY{p}{(}\PY{n}{k} \PY{o}{\PYZhy{}} \PY{l+m+mi}{1}\PY{p}{)}
        \PY{n}{x5} \PY{o}{=} \PY{n}{problema\PYZus{}3\PYZus{}calcular\PYZus{}u}\PY{p}{(}\PY{n}{k} \PY{o}{\PYZhy{}} \PY{l+m+mi}{2}\PY{p}{)}
        \PY{n}{y} \PY{o}{=} \PY{n}{problema\PYZus{}3\PYZus{}calcular\PYZus{}g}\PY{p}{(}\PY{n}{x1}\PY{p}{,} \PY{n}{x2}\PY{p}{,} \PY{n}{x3}\PY{p}{,} \PY{n}{x4}\PY{p}{,} \PY{n}{x5}\PY{p}{)}
        \PY{k}{return} \PY{n}{y}\PY{p}{,} \PY{n}{x1}\PY{p}{,} \PY{n}{x2}\PY{p}{,} \PY{n}{x3}\PY{p}{,} \PY{n}{x4}\PY{p}{,} \PY{n}{x5}


\PY{k}{def} \PY{n+nf}{problema\PYZus{}3\PYZus{}criar\PYZus{}arquivos}\PY{p}{(}\PY{p}{)}\PY{p}{:}
    \PY{n}{xt} \PY{o}{=} \PY{p}{[}\PY{p}{]}
    \PY{n}{ydt} \PY{o}{=} \PY{p}{[}\PY{p}{]}
    \PY{k}{for} \PY{n}{k} \PY{o+ow}{in} \PY{n+nb}{range}\PY{p}{(}\PY{l+m+mi}{0}\PY{p}{,} \PY{l+m+mi}{5001}\PY{p}{)}\PY{p}{:}
        \PY{n}{y}\PY{p}{,} \PY{n}{x1}\PY{p}{,} \PY{n}{x2}\PY{p}{,} \PY{n}{x3}\PY{p}{,} \PY{n}{x4}\PY{p}{,} \PY{n}{x5} \PY{o}{=} \PY{n}{problema\PYZus{}3\PYZus{}calcular\PYZus{}y}\PY{p}{(}\PY{n}{k}\PY{p}{,} \PY{n}{xt}\PY{p}{,} \PY{n}{ydt}\PY{p}{)}
        \PY{n}{xt}\PY{o}{.}\PY{n}{append}\PY{p}{(}\PY{p}{[}\PY{n}{x1}\PY{p}{,} \PY{n}{x2}\PY{p}{,} \PY{n}{x3}\PY{p}{,} \PY{n}{x4}\PY{p}{,} \PY{n}{x5}\PY{p}{]}\PY{p}{)}
        \PY{n}{ydt}\PY{o}{.}\PY{n}{append}\PY{p}{(}\PY{p}{[}\PY{n}{y}\PY{p}{]}\PY{p}{)}
    \PY{n}{xt}\PY{p}{,} \PY{n}{ydt} \PY{o}{=} \PY{n}{embaralhar\PYZus{}dados}\PY{p}{(}\PY{n}{xt}\PY{p}{,} \PY{n}{ydt}\PY{p}{)}
    \PY{n}{criar\PYZus{}arquivo}\PY{p}{(}\PY{l+s+s1}{\PYZsq{}}\PY{l+s+s1}{xte3.csv}\PY{l+s+s1}{\PYZsq{}}\PY{p}{,} \PY{n}{xt}\PY{p}{)}
    \PY{n}{criar\PYZus{}arquivo}\PY{p}{(}\PY{l+s+s1}{\PYZsq{}}\PY{l+s+s1}{ydte3.csv}\PY{l+s+s1}{\PYZsq{}}\PY{p}{,} \PY{n}{ydt}\PY{p}{)}

    \PY{n}{xv} \PY{o}{=} \PY{p}{[}\PY{p}{]}
    \PY{n}{ydv} \PY{o}{=} \PY{p}{[}\PY{p}{]}
    \PY{k}{for} \PY{n}{k} \PY{o+ow}{in} \PY{n+nb}{range}\PY{p}{(}\PY{l+m+mi}{0}\PY{p}{,} \PY{l+m+mi}{10001}\PY{p}{)}\PY{p}{:}
        \PY{n}{y}\PY{p}{,} \PY{n}{x1}\PY{p}{,} \PY{n}{x2}\PY{p}{,} \PY{n}{x3}\PY{p}{,} \PY{n}{x4}\PY{p}{,} \PY{n}{x5} \PY{o}{=} \PY{n}{problema\PYZus{}3\PYZus{}calcular\PYZus{}y}\PY{p}{(}\PY{n}{k}\PY{p}{,} \PY{n}{xv}\PY{p}{,} \PY{n}{ydv}\PY{p}{)}
        \PY{n}{xv}\PY{o}{.}\PY{n}{append}\PY{p}{(}\PY{p}{[}\PY{n}{x1}\PY{p}{,} \PY{n}{x2}\PY{p}{,} \PY{n}{x3}\PY{p}{,} \PY{n}{x4}\PY{p}{,} \PY{n}{x5}\PY{p}{]}\PY{p}{)}
        \PY{n}{ydv}\PY{o}{.}\PY{n}{append}\PY{p}{(}\PY{p}{[}\PY{n}{y}\PY{p}{]}\PY{p}{)}
    \PY{n}{criar\PYZus{}arquivo}\PY{p}{(}\PY{l+s+s1}{\PYZsq{}}\PY{l+s+s1}{xve3.csv}\PY{l+s+s1}{\PYZsq{}}\PY{p}{,} \PY{n}{xv}\PY{p}{)}
    \PY{n}{criar\PYZus{}arquivo}\PY{p}{(}\PY{l+s+s1}{\PYZsq{}}\PY{l+s+s1}{ydve3.csv}\PY{l+s+s1}{\PYZsq{}}\PY{p}{,} \PY{n}{ydv}\PY{p}{)}
\end{Verbatim}
\end{tcolorbox}

    \hypertarget{carregamento-dos-dados-de-treino-e-validauxe7uxe3o}{%
\paragraph{Carregamento dos dados de treino e
validação}\label{carregamento-dos-dados-de-treino-e-validauxe7uxe3o}}

    \begin{tcolorbox}[breakable, size=fbox, boxrule=1pt, pad at break*=1mm,colback=cellbackground, colframe=cellborder]
\prompt{In}{incolor}{31}{\boxspacing}
\begin{Verbatim}[commandchars=\\\{\}]
\PY{n}{xt} \PY{o}{=} \PY{n}{pd}\PY{o}{.}\PY{n}{read\PYZus{}csv}\PY{p}{(}\PY{l+s+s1}{\PYZsq{}}\PY{l+s+s1}{xte3.csv}\PY{l+s+s1}{\PYZsq{}}\PY{p}{)}\PY{o}{.}\PY{n}{values}\PY{o}{.}\PY{n}{tolist}\PY{p}{(}\PY{p}{)}
\PY{n}{ydt} \PY{o}{=} \PY{n}{pd}\PY{o}{.}\PY{n}{read\PYZus{}csv}\PY{p}{(}\PY{l+s+s1}{\PYZsq{}}\PY{l+s+s1}{ydte3.csv}\PY{l+s+s1}{\PYZsq{}}\PY{p}{)}\PY{o}{.}\PY{n}{iloc}\PY{p}{[}\PY{p}{:}\PY{p}{,} \PY{l+m+mi}{0}\PY{p}{]}\PY{o}{.}\PY{n}{tolist}\PY{p}{(}\PY{p}{)}
\PY{n}{xv} \PY{o}{=} \PY{n}{pd}\PY{o}{.}\PY{n}{read\PYZus{}csv}\PY{p}{(}\PY{l+s+s1}{\PYZsq{}}\PY{l+s+s1}{xve3.csv}\PY{l+s+s1}{\PYZsq{}}\PY{p}{)}\PY{o}{.}\PY{n}{values}\PY{o}{.}\PY{n}{tolist}\PY{p}{(}\PY{p}{)}
\PY{n}{ydv} \PY{o}{=} \PY{n}{pd}\PY{o}{.}\PY{n}{read\PYZus{}csv}\PY{p}{(}\PY{l+s+s1}{\PYZsq{}}\PY{l+s+s1}{ydve3.csv}\PY{l+s+s1}{\PYZsq{}}\PY{p}{)}\PY{o}{.}\PY{n}{iloc}\PY{p}{[}\PY{p}{:}\PY{p}{,} \PY{l+m+mi}{0}\PY{p}{]}\PY{o}{.}\PY{n}{tolist}\PY{p}{(}\PY{p}{)}
\end{Verbatim}
\end{tcolorbox}

    \hypertarget{resultado-desejado}{%
\paragraph{Resultado desejado}\label{resultado-desejado}}

    \begin{tcolorbox}[breakable, size=fbox, boxrule=1pt, pad at break*=1mm,colback=cellbackground, colframe=cellborder]
\prompt{In}{incolor}{32}{\boxspacing}
\begin{Verbatim}[commandchars=\\\{\}]
\PY{n}{label\PYZus{}y\PYZus{}validacao} \PY{o}{=} \PY{l+s+s2}{\PYZdq{}}\PY{l+s+s2}{\PYZdl{}y(k+1) = g[y(k), y(k\PYZhy{}1), y(k\PYZhy{}2), u(k), u(k\PYZhy{}1)]\PYZdl{}}\PY{l+s+s2}{\PYZdq{}}
\PY{n}{exibir\PYZus{}resultado\PYZus{}desejado}\PY{p}{(}\PY{n}{xv}\PY{p}{,} \PY{n}{ydv}\PY{p}{,} \PY{n}{label\PYZus{}y\PYZus{}validacao}\PY{p}{)}
\end{Verbatim}
\end{tcolorbox}

    \begin{center}
    \adjustimage{max size={0.9\linewidth}{0.9\paperheight}}{ExercicioComputacionalParte2_files/ExercicioComputacionalParte2_67_0.png}
    \end{center}
    { \hspace*{\fill} \\}
    
    \hypertarget{resumo-dos-dados-de-treino-e-validauxe7uxe3o}{%
\subsubsection{Resumo dos dados de treino e
validação}\label{resumo-dos-dados-de-treino-e-validauxe7uxe3o}}

    \begin{tcolorbox}[breakable, size=fbox, boxrule=1pt, pad at break*=1mm,colback=cellbackground, colframe=cellborder]
\prompt{In}{incolor}{33}{\boxspacing}
\begin{Verbatim}[commandchars=\\\{\}]
\PY{n}{descrever\PYZus{}dados}\PY{p}{(}\PY{n}{xt}\PY{p}{,} \PY{n}{ydt}\PY{p}{,} \PY{n}{xv}\PY{p}{,} \PY{n}{ydv}\PY{p}{)}
\end{Verbatim}
\end{tcolorbox}

    \begin{Verbatim}[commandchars=\\\{\}]

===Dataset xt===
             0         1         2         3         4
0    -0.446678 -0.435928 -0.429134 -0.680864 -0.642804
1     0.021002  0.055787  0.078425 -0.025314  0.021197
2    -0.178104 -0.208557 -0.245899 -0.168059 -0.195192
3    -0.287780 -0.331321 -0.373264 -0.283230 -0.340765
4     0.244241  0.227010  0.208546  0.297042  0.272952
{\ldots}        {\ldots}       {\ldots}       {\ldots}       {\ldots}       {\ldots}
4995  0.383368  0.379683  0.379485  0.612949  0.595618
4996 -0.374514 -0.345878 -0.307980 -0.512466 -0.467202
4997  0.293578  0.287271  0.287305  0.380735  0.363944
4998  0.434077  0.434982  0.435461  0.951010  0.975854
4999  0.432227  0.428600  0.422905  0.914215  0.868656

[5000 rows x 5 columns]

===Dataset yt===
             0
0    -0.459800
1    -0.025145
2    -0.156405
3    -0.245899
4     0.260261
{\ldots}        {\ldots}
4995  0.389898
4996 -0.395474
4997  0.305050
4998  0.432518
4999  0.434292

[5000 rows x 1 columns]

===Dataset xv===
             0         1         2         3         4
0     0.000000  0.000000  0.000000  0.000000  0.000000
1     0.000000  0.000000  0.000000  0.025130  0.000000
2     0.025114  0.000000  0.000000  0.050244  0.025130
3     0.050118  0.025114  0.000000  0.075327  0.050244
4     0.074902  0.050118  0.025114  0.100362  0.075327
{\ldots}        {\ldots}       {\ldots}       {\ldots}       {\ldots}       {\ldots}
9995 -0.278551 -0.291155 -0.295093 -0.290478 -0.319786
9996 -0.256424 -0.278551 -0.291155 -0.249155 -0.290478
9997 -0.224053 -0.256424 -0.278551 -0.197171 -0.249155
9998 -0.181169 -0.224053 -0.256424 -0.136546 -0.197171
9999 -0.128297 -0.181169 -0.224053 -0.069842 -0.136546

[10000 rows x 5 columns]

===Dataset yv===
             0
0     0.000000
1     0.025114
2     0.050118
3     0.074902
4     0.099292
{\ldots}        {\ldots}
9995 -0.256424
9996 -0.224053
9997 -0.181169
9998 -0.128297
9999 -0.067021

[10000 rows x 1 columns]
    \end{Verbatim}

    \hypertarget{estruturas-implementadas}{%
\subsubsection{Estruturas
implementadas}\label{estruturas-implementadas}}

    \hypertarget{paruxe2metros-gerais}{%
\paragraph{Parâmetros gerais}\label{paruxe2metros-gerais}}

    \begin{tcolorbox}[breakable, size=fbox, boxrule=1pt, pad at break*=1mm,colback=cellbackground, colframe=cellborder]
\prompt{In}{incolor}{34}{\boxspacing}
\begin{Verbatim}[commandchars=\\\{\}]
\PY{n}{n\PYZus{}epoca} \PY{o}{=} \PY{l+m+mi}{20}
\end{Verbatim}
\end{tcolorbox}

    \hypertarget{estrutura-anfis}{%
\paragraph{Estrutura ANFIS}\label{estrutura-anfis}}

Parâmetro

    \begin{tcolorbox}[breakable, size=fbox, boxrule=1pt, pad at break*=1mm,colback=cellbackground, colframe=cellborder]
\prompt{In}{incolor}{35}{\boxspacing}
\begin{Verbatim}[commandchars=\\\{\}]
\PY{n}{m} \PY{o}{=} \PY{l+m+mi}{5}
\PY{n}{alfa} \PY{o}{=} \PY{l+m+mf}{0.2}
\PY{n}{anfis\PYZus{}problema\PYZus{}3} \PY{o}{=} \PY{n}{ANFIS}\PY{p}{(}\PY{n}{n\PYZus{}epoca}\PY{p}{,} \PY{n}{xt}\PY{p}{,} \PY{n}{ydt}\PY{p}{,} \PY{n}{xv}\PY{p}{,} \PY{n}{ydv}\PY{p}{,} \PY{n}{m}\PY{p}{,} \PY{n}{label\PYZus{}y\PYZus{}validacao}\PY{p}{,} \PY{n}{alfa}\PY{p}{)}
\end{Verbatim}
\end{tcolorbox}

    Demonstração de estado inicial

    \begin{tcolorbox}[breakable, size=fbox, boxrule=1pt, pad at break*=1mm,colback=cellbackground, colframe=cellborder]
\prompt{In}{incolor}{36}{\boxspacing}
\begin{Verbatim}[commandchars=\\\{\}]
\PY{n}{anfis\PYZus{}problema\PYZus{}3}\PY{o}{.}\PY{n}{exibir\PYZus{}resultado\PYZus{}validacao}\PY{p}{(}\PY{p}{)}
\end{Verbatim}
\end{tcolorbox}

    \begin{center}
    \adjustimage{max size={0.9\linewidth}{0.9\paperheight}}{ExercicioComputacionalParte2_files/ExercicioComputacionalParte2_76_0.png}
    \end{center}
    { \hspace*{\fill} \\}
    
    \begin{Verbatim}[commandchars=\\\{\}]
Root mean squared error (RMSE):  0.7277
Mean absolute percentage error (MAPE): 71679.5019\%
    \end{Verbatim}

    Demonstração de treinamento

    \begin{tcolorbox}[breakable, size=fbox, boxrule=1pt, pad at break*=1mm,colback=cellbackground, colframe=cellborder]
\prompt{In}{incolor}{37}{\boxspacing}
\begin{Verbatim}[commandchars=\\\{\}]
\PY{n}{anfis\PYZus{}problema\PYZus{}3}\PY{o}{.}\PY{n}{treinar\PYZus{}gradiente}\PY{p}{(}\PY{n}{plota\PYZus{}resultado\PYZus{}epocas}\PY{o}{=}\PY{k+kc}{True}\PY{p}{)}
\end{Verbatim}
\end{tcolorbox}

    \begin{center}
    \adjustimage{max size={0.9\linewidth}{0.9\paperheight}}{ExercicioComputacionalParte2_files/ExercicioComputacionalParte2_78_0.png}
    \end{center}
    { \hspace*{\fill} \\}
    
    \begin{center}
    \adjustimage{max size={0.9\linewidth}{0.9\paperheight}}{ExercicioComputacionalParte2_files/ExercicioComputacionalParte2_78_1.png}
    \end{center}
    { \hspace*{\fill} \\}
    
    Demonstração de resultado final

    \begin{tcolorbox}[breakable, size=fbox, boxrule=1pt, pad at break*=1mm,colback=cellbackground, colframe=cellborder]
\prompt{In}{incolor}{38}{\boxspacing}
\begin{Verbatim}[commandchars=\\\{\}]
\PY{n}{anfis\PYZus{}problema\PYZus{}3}\PY{o}{.}\PY{n}{exibir\PYZus{}resultado\PYZus{}validacao}\PY{p}{(}\PY{p}{)}
\end{Verbatim}
\end{tcolorbox}

    \begin{center}
    \adjustimage{max size={0.9\linewidth}{0.9\paperheight}}{ExercicioComputacionalParte2_files/ExercicioComputacionalParte2_80_0.png}
    \end{center}
    { \hspace*{\fill} \\}
    
    \begin{Verbatim}[commandchars=\\\{\}]
Root mean squared error (RMSE):  0.0026
Mean absolute percentage error (MAPE): 174.8272\%
    \end{Verbatim}

    Resultado médio para 10 treinos

    \begin{tcolorbox}[breakable, size=fbox, boxrule=1pt, pad at break*=1mm,colback=cellbackground, colframe=cellborder]
\prompt{In}{incolor}{39}{\boxspacing}
\begin{Verbatim}[commandchars=\\\{\}]
\PY{n}{anfis\PYZus{}problema\PYZus{}3}\PY{o}{.}\PY{n}{exibir\PYZus{}resultado\PYZus{}validacao\PYZus{}multiplos\PYZus{}treinos}\PY{p}{(}\PY{l+m+mi}{10}\PY{p}{)}
\end{Verbatim}
\end{tcolorbox}

    \begin{center}
    \adjustimage{max size={0.9\linewidth}{0.9\paperheight}}{ExercicioComputacionalParte2_files/ExercicioComputacionalParte2_82_0.png}
    \end{center}
    { \hspace*{\fill} \\}
    
    \begin{Verbatim}[commandchars=\\\{\}]
Média do RMSE:  0.0058
Desvio padrão do RMSE:  0.0024
    \end{Verbatim}

    \begin{center}
    \adjustimage{max size={0.9\linewidth}{0.9\paperheight}}{ExercicioComputacionalParte2_files/ExercicioComputacionalParte2_82_2.png}
    \end{center}
    { \hspace*{\fill} \\}
    
    \begin{Verbatim}[commandchars=\\\{\}]
Média do MAPE: 430.1932\%
Desvio padrão do MAPE: 276.0701\%
    \end{Verbatim}

    \hypertarget{estrutura-nfn}{%
\paragraph{Estrutura NFN}\label{estrutura-nfn}}

Parâmetro

    \begin{tcolorbox}[breakable, size=fbox, boxrule=1pt, pad at break*=1mm,colback=cellbackground, colframe=cellborder]
\prompt{In}{incolor}{40}{\boxspacing}
\begin{Verbatim}[commandchars=\\\{\}]
\PY{n}{m} \PY{o}{=} \PY{l+m+mi}{100}
\PY{n}{alfa} \PY{o}{=} \PY{l+m+mf}{0.4}
\PY{n}{nfn\PYZus{}problema\PYZus{}3} \PY{o}{=} \PY{n}{NFN}\PY{p}{(}\PY{n}{n\PYZus{}epoca}\PY{p}{,} \PY{n}{xt}\PY{p}{,} \PY{n}{ydt}\PY{p}{,} \PY{n}{xv}\PY{p}{,} \PY{n}{ydv}\PY{p}{,} \PY{n}{m}\PY{p}{,} \PY{n}{label\PYZus{}y\PYZus{}validacao}\PY{p}{,} \PY{n}{alfa}\PY{p}{)}
\end{Verbatim}
\end{tcolorbox}

    Demonstração de estado inicial

    \begin{tcolorbox}[breakable, size=fbox, boxrule=1pt, pad at break*=1mm,colback=cellbackground, colframe=cellborder]
\prompt{In}{incolor}{41}{\boxspacing}
\begin{Verbatim}[commandchars=\\\{\}]
\PY{n}{nfn\PYZus{}problema\PYZus{}3}\PY{o}{.}\PY{n}{exibir\PYZus{}resultado\PYZus{}validacao}\PY{p}{(}\PY{p}{)}
\end{Verbatim}
\end{tcolorbox}

    \begin{center}
    \adjustimage{max size={0.9\linewidth}{0.9\paperheight}}{ExercicioComputacionalParte2_files/ExercicioComputacionalParte2_86_0.png}
    \end{center}
    { \hspace*{\fill} \\}
    
    \begin{Verbatim}[commandchars=\\\{\}]
Root mean squared error (RMSE):  0.9538
Mean absolute percentage error (MAPE): 86086.7083\%
    \end{Verbatim}

    Demonstração de treinamento

    \begin{tcolorbox}[breakable, size=fbox, boxrule=1pt, pad at break*=1mm,colback=cellbackground, colframe=cellborder]
\prompt{In}{incolor}{42}{\boxspacing}
\begin{Verbatim}[commandchars=\\\{\}]
\PY{n}{nfn\PYZus{}problema\PYZus{}3}\PY{o}{.}\PY{n}{treinar\PYZus{}gradiente}\PY{p}{(}\PY{n}{plota\PYZus{}resultado\PYZus{}epocas}\PY{o}{=}\PY{k+kc}{True}\PY{p}{)}
\end{Verbatim}
\end{tcolorbox}

    \begin{center}
    \adjustimage{max size={0.9\linewidth}{0.9\paperheight}}{ExercicioComputacionalParte2_files/ExercicioComputacionalParte2_88_0.png}
    \end{center}
    { \hspace*{\fill} \\}
    
    \begin{center}
    \adjustimage{max size={0.9\linewidth}{0.9\paperheight}}{ExercicioComputacionalParte2_files/ExercicioComputacionalParte2_88_1.png}
    \end{center}
    { \hspace*{\fill} \\}
    
    Demonstração de resultado final

    \begin{tcolorbox}[breakable, size=fbox, boxrule=1pt, pad at break*=1mm,colback=cellbackground, colframe=cellborder]
\prompt{In}{incolor}{43}{\boxspacing}
\begin{Verbatim}[commandchars=\\\{\}]
\PY{n}{nfn\PYZus{}problema\PYZus{}3}\PY{o}{.}\PY{n}{exibir\PYZus{}resultado\PYZus{}validacao}\PY{p}{(}\PY{p}{)}
\end{Verbatim}
\end{tcolorbox}

    \begin{center}
    \adjustimage{max size={0.9\linewidth}{0.9\paperheight}}{ExercicioComputacionalParte2_files/ExercicioComputacionalParte2_90_0.png}
    \end{center}
    { \hspace*{\fill} \\}
    
    \begin{Verbatim}[commandchars=\\\{\}]
Root mean squared error (RMSE):  0.0061
Mean absolute percentage error (MAPE): 107.0212\%
    \end{Verbatim}

    Resultado médio para 10 treinos

    \begin{tcolorbox}[breakable, size=fbox, boxrule=1pt, pad at break*=1mm,colback=cellbackground, colframe=cellborder]
\prompt{In}{incolor}{44}{\boxspacing}
\begin{Verbatim}[commandchars=\\\{\}]
\PY{n}{nfn\PYZus{}problema\PYZus{}3}\PY{o}{.}\PY{n}{exibir\PYZus{}resultado\PYZus{}validacao\PYZus{}multiplos\PYZus{}treinos}\PY{p}{(}\PY{l+m+mi}{10}\PY{p}{)}
\end{Verbatim}
\end{tcolorbox}

    \begin{center}
    \adjustimage{max size={0.9\linewidth}{0.9\paperheight}}{ExercicioComputacionalParte2_files/ExercicioComputacionalParte2_92_0.png}
    \end{center}
    { \hspace*{\fill} \\}
    
    \begin{Verbatim}[commandchars=\\\{\}]
Média do RMSE:  0.0072
Desvio padrão do RMSE:  0.0008
    \end{Verbatim}

    \begin{center}
    \adjustimage{max size={0.9\linewidth}{0.9\paperheight}}{ExercicioComputacionalParte2_files/ExercicioComputacionalParte2_92_2.png}
    \end{center}
    { \hspace*{\fill} \\}
    
    \begin{Verbatim}[commandchars=\\\{\}]
Média do MAPE: 353.4722\%
Desvio padrão do MAPE: 154.0003\%
    \end{Verbatim}

    \hypertarget{problema-4-exemplo-4-do-livro-texto-previsuxe3o-de-suxe9ries-temporais}{%
\subsection{Problema 4: Exemplo 4 do livro texto (previsão de séries
temporais)}\label{problema-4-exemplo-4-do-livro-texto-previsuxe3o-de-suxe9ries-temporais}}

Este problema é o exemplo 4 do capítulo 12 do livro texto da disciplina.
Esse consiste em aproximação de uma série temporal caótica descrita pela
seguinte função:

\(\dot{x} = \frac{0.2x(t - \tau)}{1+x^{10}(t-\tau)} - 0.1x(t)\)

As entradas desse problema são variáveis x(t), x(t-6), x(t-12) e x(t-18)
e saída x(t+6).

\([x(t-18), x(t-12), x(t-6), x(t); x(t+6)]\)

utilizar os arquivos disponibilizados.

\hypertarget{criauxe7uxe3o-dos-arquivos-de-treino-e-validauxe7uxe3o}{%
\subsubsection{Criação dos arquivos de treino e
validação}\label{criauxe7uxe3o-dos-arquivos-de-treino-e-validauxe7uxe3o}}

    \begin{tcolorbox}[breakable, size=fbox, boxrule=1pt, pad at break*=1mm,colback=cellbackground, colframe=cellborder]
\prompt{In}{incolor}{45}{\boxspacing}
\begin{Verbatim}[commandchars=\\\{\}]
\PY{k}{def} \PY{n+nf}{problema\PYZus{}4\PYZus{}criar\PYZus{}arquivos}\PY{p}{(}\PY{p}{)}\PY{p}{:}
    \PY{n}{dados\PYZus{}mg} \PY{o}{=} \PY{n}{pd}\PY{o}{.}\PY{n}{read\PYZus{}csv}\PY{p}{(}\PY{l+s+s2}{\PYZdq{}}\PY{l+s+s2}{mgdata.dat.txt}\PY{l+s+s2}{\PYZdq{}}\PY{p}{,} \PY{n}{sep}\PY{o}{=}\PY{l+s+s2}{\PYZdq{}}\PY{l+s+s2}{ }\PY{l+s+s2}{\PYZdq{}}\PY{p}{,} \PY{n}{header}\PY{o}{=}\PY{k+kc}{None}\PY{p}{,} \PY{n}{names}\PY{o}{=}\PY{p}{[}\PY{l+s+s1}{\PYZsq{}}\PY{l+s+s1}{indice}\PY{l+s+s1}{\PYZsq{}}\PY{p}{,} \PY{l+s+s1}{\PYZsq{}}\PY{l+s+s1}{valor}\PY{l+s+s1}{\PYZsq{}}\PY{p}{]}\PY{p}{)}

    \PY{n}{x1} \PY{o}{=} \PY{n}{atrasar}\PY{p}{(}\PY{n}{dados\PYZus{}mg}\PY{o}{.}\PY{n}{valor}\PY{p}{,} \PY{l+m+mi}{18}\PY{p}{)}
    \PY{n}{x2} \PY{o}{=} \PY{n}{atrasar}\PY{p}{(}\PY{n}{dados\PYZus{}mg}\PY{o}{.}\PY{n}{valor}\PY{p}{,} \PY{l+m+mi}{12}\PY{p}{)}
    \PY{n}{x3} \PY{o}{=} \PY{n}{atrasar}\PY{p}{(}\PY{n}{dados\PYZus{}mg}\PY{o}{.}\PY{n}{valor}\PY{p}{,} \PY{l+m+mi}{6}\PY{p}{)}
    \PY{n}{x4} \PY{o}{=} \PY{n}{atrasar}\PY{p}{(}\PY{n}{dados\PYZus{}mg}\PY{o}{.}\PY{n}{valor}\PY{p}{,} \PY{l+m+mi}{0}\PY{p}{)}
    \PY{n}{y} \PY{o}{=} \PY{n}{avancar}\PY{p}{(}\PY{n}{dados\PYZus{}mg}\PY{o}{.}\PY{n}{valor}\PY{p}{,} \PY{l+m+mi}{6}\PY{p}{)}

    \PY{n}{x1} \PY{o}{=} \PY{n}{ajustar}\PY{p}{(}\PY{n}{x1}\PY{p}{,} \PY{l+m+mi}{18}\PY{p}{,} \PY{l+m+mi}{6}\PY{p}{)}
    \PY{n}{x2} \PY{o}{=} \PY{n}{ajustar}\PY{p}{(}\PY{n}{x2}\PY{p}{,} \PY{l+m+mi}{18}\PY{p}{,} \PY{l+m+mi}{6}\PY{p}{)}
    \PY{n}{x3} \PY{o}{=} \PY{n}{ajustar}\PY{p}{(}\PY{n}{x3}\PY{p}{,} \PY{l+m+mi}{18}\PY{p}{,} \PY{l+m+mi}{6}\PY{p}{)}
    \PY{n}{x4} \PY{o}{=} \PY{n}{ajustar}\PY{p}{(}\PY{n}{x4}\PY{p}{,} \PY{l+m+mi}{18}\PY{p}{,} \PY{l+m+mi}{6}\PY{p}{)}
    \PY{n}{y} \PY{o}{=} \PY{n}{ajustar}\PY{p}{(}\PY{n}{y}\PY{p}{,} \PY{l+m+mi}{18}\PY{p}{,} \PY{l+m+mi}{6}\PY{p}{)}

    \PY{n}{x} \PY{o}{=} \PY{n}{np}\PY{o}{.}\PY{n}{array}\PY{p}{(}\PY{p}{[}\PY{n}{x1}\PY{p}{,} \PY{n}{x2}\PY{p}{,} \PY{n}{x3}\PY{p}{,} \PY{n}{x4}\PY{p}{]}\PY{p}{)}\PY{o}{.}\PY{n}{T}

    \PY{n}{limite\PYZus{}amostra\PYZus{}treino} \PY{o}{=} \PY{l+m+mi}{301}

    \PY{n}{xt} \PY{o}{=} \PY{n}{x}\PY{p}{[}\PY{p}{:}\PY{n}{limite\PYZus{}amostra\PYZus{}treino}\PY{p}{]}
    \PY{n}{ydt} \PY{o}{=} \PY{n}{y}\PY{p}{[}\PY{p}{:}\PY{n}{limite\PYZus{}amostra\PYZus{}treino}\PY{p}{]}
    \PY{n}{xt}\PY{p}{,} \PY{n}{ydt} \PY{o}{=} \PY{n}{embaralhar\PYZus{}dados}\PY{p}{(}\PY{n}{xt}\PY{p}{,} \PY{n}{ydt}\PY{p}{)}

    \PY{n}{xv} \PY{o}{=} \PY{n}{x}\PY{p}{[}\PY{n}{limite\PYZus{}amostra\PYZus{}treino}\PY{p}{:}\PY{p}{]}
    \PY{n}{ydv} \PY{o}{=} \PY{n}{y}\PY{p}{[}\PY{n}{limite\PYZus{}amostra\PYZus{}treino}\PY{p}{:}\PY{p}{]}

    \PY{n}{criar\PYZus{}arquivo}\PY{p}{(}\PY{l+s+s1}{\PYZsq{}}\PY{l+s+s1}{xte4.csv}\PY{l+s+s1}{\PYZsq{}}\PY{p}{,} \PY{n}{xt}\PY{p}{)}
    \PY{n}{criar\PYZus{}arquivo}\PY{p}{(}\PY{l+s+s1}{\PYZsq{}}\PY{l+s+s1}{ydte4.csv}\PY{l+s+s1}{\PYZsq{}}\PY{p}{,} \PY{n}{ydt}\PY{p}{)}
    \PY{n}{criar\PYZus{}arquivo}\PY{p}{(}\PY{l+s+s1}{\PYZsq{}}\PY{l+s+s1}{xve4.csv}\PY{l+s+s1}{\PYZsq{}}\PY{p}{,} \PY{n}{xv}\PY{p}{)}
    \PY{n}{criar\PYZus{}arquivo}\PY{p}{(}\PY{l+s+s1}{\PYZsq{}}\PY{l+s+s1}{ydve4.csv}\PY{l+s+s1}{\PYZsq{}}\PY{p}{,} \PY{n}{ydv}\PY{p}{)}
\end{Verbatim}
\end{tcolorbox}

    \hypertarget{carregamento-dos-dados-de-treino-e-validauxe7uxe3o}{%
\subsubsection{Carregamento dos dados de treino e
validação}\label{carregamento-dos-dados-de-treino-e-validauxe7uxe3o}}

    \begin{tcolorbox}[breakable, size=fbox, boxrule=1pt, pad at break*=1mm,colback=cellbackground, colframe=cellborder]
\prompt{In}{incolor}{46}{\boxspacing}
\begin{Verbatim}[commandchars=\\\{\}]
\PY{n}{xt} \PY{o}{=} \PY{n}{pd}\PY{o}{.}\PY{n}{read\PYZus{}csv}\PY{p}{(}\PY{l+s+s1}{\PYZsq{}}\PY{l+s+s1}{xte4.csv}\PY{l+s+s1}{\PYZsq{}}\PY{p}{)}\PY{o}{.}\PY{n}{values}\PY{o}{.}\PY{n}{tolist}\PY{p}{(}\PY{p}{)}
\PY{n}{ydt} \PY{o}{=} \PY{n}{pd}\PY{o}{.}\PY{n}{read\PYZus{}csv}\PY{p}{(}\PY{l+s+s1}{\PYZsq{}}\PY{l+s+s1}{ydte4.csv}\PY{l+s+s1}{\PYZsq{}}\PY{p}{)}\PY{o}{.}\PY{n}{iloc}\PY{p}{[}\PY{p}{:}\PY{p}{,} \PY{l+m+mi}{0}\PY{p}{]}\PY{o}{.}\PY{n}{tolist}\PY{p}{(}\PY{p}{)}
\PY{n}{xv} \PY{o}{=} \PY{n}{pd}\PY{o}{.}\PY{n}{read\PYZus{}csv}\PY{p}{(}\PY{l+s+s1}{\PYZsq{}}\PY{l+s+s1}{xve4.csv}\PY{l+s+s1}{\PYZsq{}}\PY{p}{)}\PY{o}{.}\PY{n}{values}\PY{o}{.}\PY{n}{tolist}\PY{p}{(}\PY{p}{)}
\PY{n}{ydv} \PY{o}{=} \PY{n}{pd}\PY{o}{.}\PY{n}{read\PYZus{}csv}\PY{p}{(}\PY{l+s+s1}{\PYZsq{}}\PY{l+s+s1}{ydve4.csv}\PY{l+s+s1}{\PYZsq{}}\PY{p}{)}\PY{o}{.}\PY{n}{iloc}\PY{p}{[}\PY{p}{:}\PY{p}{,} \PY{l+m+mi}{0}\PY{p}{]}\PY{o}{.}\PY{n}{tolist}\PY{p}{(}\PY{p}{)}
\end{Verbatim}
\end{tcolorbox}

    \hypertarget{resultado-desejado}{%
\subsubsection{Resultado desejado}\label{resultado-desejado}}

    \begin{tcolorbox}[breakable, size=fbox, boxrule=1pt, pad at break*=1mm,colback=cellbackground, colframe=cellborder]
\prompt{In}{incolor}{47}{\boxspacing}
\begin{Verbatim}[commandchars=\\\{\}]
\PY{n}{label\PYZus{}y\PYZus{}validacao} \PY{o}{=} \PY{l+s+s2}{\PYZdq{}}\PY{l+s+s2}{Série Mackey\PYZhy{}Glass}\PY{l+s+s2}{\PYZdq{}}
\PY{n}{exibir\PYZus{}resultado\PYZus{}desejado}\PY{p}{(}\PY{n}{xv}\PY{p}{,} \PY{n}{ydv}\PY{p}{,} \PY{n}{label\PYZus{}y\PYZus{}validacao}\PY{p}{)}
\end{Verbatim}
\end{tcolorbox}

    \begin{center}
    \adjustimage{max size={0.9\linewidth}{0.9\paperheight}}{ExercicioComputacionalParte2_files/ExercicioComputacionalParte2_98_0.png}
    \end{center}
    { \hspace*{\fill} \\}
    
    \hypertarget{resumo-dos-dados-de-treino-e-validauxe7uxe3o}{%
\subsubsection{Resumo dos dados de treino e
validação}\label{resumo-dos-dados-de-treino-e-validauxe7uxe3o}}

    \begin{tcolorbox}[breakable, size=fbox, boxrule=1pt, pad at break*=1mm,colback=cellbackground, colframe=cellborder]
\prompt{In}{incolor}{48}{\boxspacing}
\begin{Verbatim}[commandchars=\\\{\}]
\PY{n}{descrever\PYZus{}dados}\PY{p}{(}\PY{n}{xt}\PY{p}{,} \PY{n}{ydt}\PY{p}{,} \PY{n}{xv}\PY{p}{,} \PY{n}{ydv}\PY{p}{)}
\end{Verbatim}
\end{tcolorbox}

    \begin{Verbatim}[commandchars=\\\{\}]

===Dataset xt===
            0         1         2         3
0    0.219220  0.655619  0.790330  0.675245
1    0.853817  0.565929  0.425606  0.741983
2    1.313696  1.084148  0.748416  0.492962
3    0.991405  1.143858  1.125572  1.121762
4    1.162239  1.041191  0.961373  0.735436
..        {\ldots}       {\ldots}       {\ldots}       {\ldots}
295  1.004313  0.776007  0.652640  0.697366
296  0.741983  0.971732  0.938761  1.064101
297  0.586417  0.937260  0.967452  0.991405
298  1.200000  0.658574  0.361433  0.246463
299  1.125572  1.121762  0.869387  0.711727

[300 rows x 4 columns]

===Dataset yt===
            0
0    0.830922
1    0.971732
2    0.474057
3    0.869387
4    0.725485
..        {\ldots}
295  1.014423
296  1.136291
297  1.143858
298  0.710732
299  0.621204

[300 rows x 1 columns]

===Dataset xv===
            0         1         2         3
0    0.942809  1.123405  1.131148  1.162401
1    0.959995  1.137060  1.132844  1.156239
2    0.988265  1.141552  1.137804  1.138207
3    1.024910  1.140181  1.145212  1.107121
4    1.064101  1.136291  1.153459  1.065472
..        {\ldots}       {\ldots}       {\ldots}       {\ldots}
870  0.951945  0.648978  0.447014  0.618967
871  0.894917  0.606960  0.435496  0.693236
872  0.840268  0.567048  0.436993  0.764479
873  0.788392  0.529843  0.455290  0.827557
874  0.739373  0.496407  0.493052  0.880008

[875 rows x 4 columns]

===Dataset yv===
            0
0    0.969583
1    0.923356
2    0.881269
3    0.844032
4    0.811458
..        {\ldots}
870  0.951448
871  0.971623
872  0.982832
873  0.986405
874  0.983860

[875 rows x 1 columns]
    \end{Verbatim}

    \hypertarget{estruturas-implementadas}{%
\subsubsection{Estruturas
implementadas}\label{estruturas-implementadas}}

    \hypertarget{paruxe2metros-gerais}{%
\paragraph{Parâmetros gerais}\label{paruxe2metros-gerais}}

    \begin{tcolorbox}[breakable, size=fbox, boxrule=1pt, pad at break*=1mm,colback=cellbackground, colframe=cellborder]
\prompt{In}{incolor}{49}{\boxspacing}
\begin{Verbatim}[commandchars=\\\{\}]
\PY{n}{n\PYZus{}epoca} \PY{o}{=} \PY{l+m+mi}{20}
\end{Verbatim}
\end{tcolorbox}

    \hypertarget{estrutura-anfis}{%
\paragraph{Estrutura ANFIS}\label{estrutura-anfis}}

Parâmetros

    \begin{tcolorbox}[breakable, size=fbox, boxrule=1pt, pad at break*=1mm,colback=cellbackground, colframe=cellborder]
\prompt{In}{incolor}{50}{\boxspacing}
\begin{Verbatim}[commandchars=\\\{\}]
\PY{n}{m} \PY{o}{=} \PY{l+m+mi}{8}
\PY{n}{alfa} \PY{o}{=} \PY{l+m+mf}{0.01}
\PY{n}{anfis\PYZus{}problema\PYZus{}4} \PY{o}{=} \PY{n}{ANFIS}\PY{p}{(}\PY{n}{n\PYZus{}epoca}\PY{p}{,} \PY{n}{xt}\PY{p}{,} \PY{n}{ydt}\PY{p}{,} \PY{n}{xv}\PY{p}{,} \PY{n}{ydv}\PY{p}{,} \PY{n}{m}\PY{p}{,} \PY{n}{label\PYZus{}y\PYZus{}validacao}\PY{p}{,} \PY{n}{alfa}\PY{p}{)}
\end{Verbatim}
\end{tcolorbox}

    Demonstração de estado inicial

    \begin{tcolorbox}[breakable, size=fbox, boxrule=1pt, pad at break*=1mm,colback=cellbackground, colframe=cellborder]
\prompt{In}{incolor}{51}{\boxspacing}
\begin{Verbatim}[commandchars=\\\{\}]
\PY{n}{anfis\PYZus{}problema\PYZus{}4}\PY{o}{.}\PY{n}{exibir\PYZus{}resultado\PYZus{}validacao}\PY{p}{(}\PY{p}{)}
\end{Verbatim}
\end{tcolorbox}

    \begin{center}
    \adjustimage{max size={0.9\linewidth}{0.9\paperheight}}{ExercicioComputacionalParte2_files/ExercicioComputacionalParte2_107_0.png}
    \end{center}
    { \hspace*{\fill} \\}
    
    \begin{Verbatim}[commandchars=\\\{\}]
Root mean squared error (RMSE):  1.1197
Mean absolute percentage error (MAPE): 132.0622\%
    \end{Verbatim}

    Demonstração de treinamento

    \begin{tcolorbox}[breakable, size=fbox, boxrule=1pt, pad at break*=1mm,colback=cellbackground, colframe=cellborder]
\prompt{In}{incolor}{52}{\boxspacing}
\begin{Verbatim}[commandchars=\\\{\}]
\PY{n}{anfis\PYZus{}problema\PYZus{}4}\PY{o}{.}\PY{n}{treinar\PYZus{}gradiente}\PY{p}{(}\PY{n}{plota\PYZus{}resultado\PYZus{}epocas}\PY{o}{=}\PY{k+kc}{True}\PY{p}{)}
\end{Verbatim}
\end{tcolorbox}

    \begin{center}
    \adjustimage{max size={0.9\linewidth}{0.9\paperheight}}{ExercicioComputacionalParte2_files/ExercicioComputacionalParte2_109_0.png}
    \end{center}
    { \hspace*{\fill} \\}
    
    \begin{center}
    \adjustimage{max size={0.9\linewidth}{0.9\paperheight}}{ExercicioComputacionalParte2_files/ExercicioComputacionalParte2_109_1.png}
    \end{center}
    { \hspace*{\fill} \\}
    
    Demonstração de resultado final

    \begin{tcolorbox}[breakable, size=fbox, boxrule=1pt, pad at break*=1mm,colback=cellbackground, colframe=cellborder]
\prompt{In}{incolor}{53}{\boxspacing}
\begin{Verbatim}[commandchars=\\\{\}]
\PY{n}{anfis\PYZus{}problema\PYZus{}4}\PY{o}{.}\PY{n}{exibir\PYZus{}resultado\PYZus{}validacao}\PY{p}{(}\PY{p}{)}
\end{Verbatim}
\end{tcolorbox}

    \begin{center}
    \adjustimage{max size={0.9\linewidth}{0.9\paperheight}}{ExercicioComputacionalParte2_files/ExercicioComputacionalParte2_111_0.png}
    \end{center}
    { \hspace*{\fill} \\}
    
    \begin{Verbatim}[commandchars=\\\{\}]
Root mean squared error (RMSE):  0.0413
Mean absolute percentage error (MAPE): 4.0194\%
    \end{Verbatim}

    Resultado médio para 10 treinos

    \begin{tcolorbox}[breakable, size=fbox, boxrule=1pt, pad at break*=1mm,colback=cellbackground, colframe=cellborder]
\prompt{In}{incolor}{54}{\boxspacing}
\begin{Verbatim}[commandchars=\\\{\}]
\PY{n}{anfis\PYZus{}problema\PYZus{}4}\PY{o}{.}\PY{n}{exibir\PYZus{}resultado\PYZus{}validacao\PYZus{}multiplos\PYZus{}treinos}\PY{p}{(}\PY{l+m+mi}{10}\PY{p}{)}
\end{Verbatim}
\end{tcolorbox}

    \begin{center}
    \adjustimage{max size={0.9\linewidth}{0.9\paperheight}}{ExercicioComputacionalParte2_files/ExercicioComputacionalParte2_113_0.png}
    \end{center}
    { \hspace*{\fill} \\}
    
    \begin{Verbatim}[commandchars=\\\{\}]
Média do RMSE:  0.0407
Desvio padrão do RMSE:  0.0137
    \end{Verbatim}

    \begin{center}
    \adjustimage{max size={0.9\linewidth}{0.9\paperheight}}{ExercicioComputacionalParte2_files/ExercicioComputacionalParte2_113_2.png}
    \end{center}
    { \hspace*{\fill} \\}
    
    \begin{Verbatim}[commandchars=\\\{\}]
Média do MAPE: 3.5082\%
Desvio padrão do MAPE: 1.0719\%
    \end{Verbatim}

    \hypertarget{estrutura-nfn}{%
\subsubsection{Estrutura NFN}\label{estrutura-nfn}}

Parâmetros

    \begin{tcolorbox}[breakable, size=fbox, boxrule=1pt, pad at break*=1mm,colback=cellbackground, colframe=cellborder]
\prompt{In}{incolor}{55}{\boxspacing}
\begin{Verbatim}[commandchars=\\\{\}]
\PY{n}{m} \PY{o}{=} \PY{l+m+mi}{8}
\PY{n}{alfa} \PY{o}{=} \PY{l+m+mf}{0.3}
\PY{n}{nfn\PYZus{}problema\PYZus{}4} \PY{o}{=} \PY{n}{NFN}\PY{p}{(}\PY{n}{n\PYZus{}epoca}\PY{p}{,} \PY{n}{xt}\PY{p}{,} \PY{n}{ydt}\PY{p}{,} \PY{n}{xv}\PY{p}{,} \PY{n}{ydv}\PY{p}{,} \PY{n}{m}\PY{p}{,} \PY{n}{label\PYZus{}y\PYZus{}validacao}\PY{p}{,} \PY{n}{alfa}\PY{p}{)}
\end{Verbatim}
\end{tcolorbox}

    Demonstração de estado inicial

    \begin{tcolorbox}[breakable, size=fbox, boxrule=1pt, pad at break*=1mm,colback=cellbackground, colframe=cellborder]
\prompt{In}{incolor}{56}{\boxspacing}
\begin{Verbatim}[commandchars=\\\{\}]
\PY{n}{nfn\PYZus{}problema\PYZus{}4}\PY{o}{.}\PY{n}{exibir\PYZus{}resultado\PYZus{}validacao}\PY{p}{(}\PY{p}{)}
\end{Verbatim}
\end{tcolorbox}

    \begin{center}
    \adjustimage{max size={0.9\linewidth}{0.9\paperheight}}{ExercicioComputacionalParte2_files/ExercicioComputacionalParte2_117_0.png}
    \end{center}
    { \hspace*{\fill} \\}
    
    \begin{Verbatim}[commandchars=\\\{\}]
Root mean squared error (RMSE):  2.3286
Mean absolute percentage error (MAPE): 272.6307\%
    \end{Verbatim}

    Demonstração de treinamento

    \begin{tcolorbox}[breakable, size=fbox, boxrule=1pt, pad at break*=1mm,colback=cellbackground, colframe=cellborder]
\prompt{In}{incolor}{57}{\boxspacing}
\begin{Verbatim}[commandchars=\\\{\}]
\PY{n}{nfn\PYZus{}problema\PYZus{}4}\PY{o}{.}\PY{n}{treinar\PYZus{}gradiente}\PY{p}{(}\PY{n}{plota\PYZus{}resultado\PYZus{}epocas}\PY{o}{=}\PY{k+kc}{True}\PY{p}{)}
\end{Verbatim}
\end{tcolorbox}

    \begin{center}
    \adjustimage{max size={0.9\linewidth}{0.9\paperheight}}{ExercicioComputacionalParte2_files/ExercicioComputacionalParte2_119_0.png}
    \end{center}
    { \hspace*{\fill} \\}
    
    \begin{center}
    \adjustimage{max size={0.9\linewidth}{0.9\paperheight}}{ExercicioComputacionalParte2_files/ExercicioComputacionalParte2_119_1.png}
    \end{center}
    { \hspace*{\fill} \\}
    
    Demonstração de resultado final

    \begin{tcolorbox}[breakable, size=fbox, boxrule=1pt, pad at break*=1mm,colback=cellbackground, colframe=cellborder]
\prompt{In}{incolor}{58}{\boxspacing}
\begin{Verbatim}[commandchars=\\\{\}]
\PY{n}{nfn\PYZus{}problema\PYZus{}4}\PY{o}{.}\PY{n}{exibir\PYZus{}resultado\PYZus{}validacao}\PY{p}{(}\PY{p}{)}
\end{Verbatim}
\end{tcolorbox}

    \begin{center}
    \adjustimage{max size={0.9\linewidth}{0.9\paperheight}}{ExercicioComputacionalParte2_files/ExercicioComputacionalParte2_121_0.png}
    \end{center}
    { \hspace*{\fill} \\}
    
    \begin{Verbatim}[commandchars=\\\{\}]
Root mean squared error (RMSE):  0.0159
Mean absolute percentage error (MAPE): 1.3533\%
    \end{Verbatim}

    Resultado médio para 10 treinos

    \begin{tcolorbox}[breakable, size=fbox, boxrule=1pt, pad at break*=1mm,colback=cellbackground, colframe=cellborder]
\prompt{In}{incolor}{59}{\boxspacing}
\begin{Verbatim}[commandchars=\\\{\}]
\PY{n}{nfn\PYZus{}problema\PYZus{}4}\PY{o}{.}\PY{n}{exibir\PYZus{}resultado\PYZus{}validacao\PYZus{}multiplos\PYZus{}treinos}\PY{p}{(}\PY{l+m+mi}{10}\PY{p}{)}
\end{Verbatim}
\end{tcolorbox}

    \begin{center}
    \adjustimage{max size={0.9\linewidth}{0.9\paperheight}}{ExercicioComputacionalParte2_files/ExercicioComputacionalParte2_123_0.png}
    \end{center}
    { \hspace*{\fill} \\}
    
    \begin{Verbatim}[commandchars=\\\{\}]
Média do RMSE:  0.0168
Desvio padrão do RMSE:  0.001
    \end{Verbatim}

    \begin{center}
    \adjustimage{max size={0.9\linewidth}{0.9\paperheight}}{ExercicioComputacionalParte2_files/ExercicioComputacionalParte2_123_2.png}
    \end{center}
    { \hspace*{\fill} \\}
    
    \begin{Verbatim}[commandchars=\\\{\}]
Média do MAPE: 1.4468\%
Desvio padrão do MAPE: 0.1101\%
    \end{Verbatim}

    \hypertarget{problema-5-escolha-um-problema-de-regressuxe3o-no-reposituxf3rio-httpsarchive.ics.uci.edumldatasets}{%
\subsection{Problema 5: Escolha um problema de regressão no repositório:
https://archive.ics.uci.edu/ml/datasets}\label{problema-5-escolha-um-problema-de-regressuxe3o-no-reposituxf3rio-httpsarchive.ics.uci.edumldatasets}}

Problema escolhido: Metro Interstate Traffic Volume Data Set
(https://archive.ics.uci.edu/ml/datasets/Metro+Interstate+Traffic+Volume)

O objetivo proposto neste problema é prever com 24 horas de antecedência
o número de pessoas que vão deslocar entre duas estações de metrô de
Minneapolis (EUA).

Para isso a seguinte fórmula para previsão de série temporal foi
experimentada:

\(y(k+24) = g[y(k),y(k-24),y(k-144)]\),

onde \(x_{1}\) representa o volume passageiros registrado na hora atual,
\(x_{2}\) representa o volume de passageiros um dia atrás e \(x_{3}\)
representa o volume de passageiros uma semana antes da data que se
pretende prever.

\hypertarget{data-set}{%
\subsubsection{Data set}\label{data-set}}

A base de dados é composta pelo volume de tráfego por hora entre 2
estações de metrô de Minneapolis. Os dados correspondem ao período de
2012 até 2018.

    \begin{tcolorbox}[breakable, size=fbox, boxrule=1pt, pad at break*=1mm,colback=cellbackground, colframe=cellborder]
\prompt{In}{incolor}{60}{\boxspacing}
\begin{Verbatim}[commandchars=\\\{\}]
\PY{n}{dados\PYZus{}metro} \PY{o}{=} \PY{n}{pd}\PY{o}{.}\PY{n}{read\PYZus{}csv}\PY{p}{(}\PY{l+s+s2}{\PYZdq{}}\PY{l+s+s2}{Metro\PYZus{}Interstate\PYZus{}Traffic\PYZus{}Volume.csv}\PY{l+s+s2}{\PYZdq{}}\PY{p}{,} \PY{n}{sep}\PY{o}{=}\PY{l+s+s2}{\PYZdq{}}\PY{l+s+s2}{,}\PY{l+s+s2}{\PYZdq{}}\PY{p}{)}

\PY{n}{plt}\PY{o}{.}\PY{n}{figure}\PY{p}{(}\PY{n}{figsize}\PY{o}{=}\PY{p}{(}\PY{l+m+mi}{8}\PY{p}{,} \PY{l+m+mi}{6}\PY{p}{)}\PY{p}{)}
\PY{n}{plt}\PY{o}{.}\PY{n}{gca}\PY{p}{(}\PY{p}{)}\PY{o}{.}\PY{n}{xaxis}\PY{o}{.}\PY{n}{set\PYZus{}major\PYZus{}locator}\PY{p}{(}\PY{n}{mdates}\PY{o}{.}\PY{n}{DayLocator}\PY{p}{(}\PY{n}{interval}\PY{o}{=}\PY{l+m+mi}{168}\PY{p}{)}\PY{p}{)}
\PY{n}{plt}\PY{o}{.}\PY{n}{plot}\PY{p}{(}\PY{n}{dados\PYZus{}metro}\PY{p}{[}\PY{p}{:}\PY{l+m+mi}{720}\PY{p}{]}\PY{o}{.}\PY{n}{date\PYZus{}time}\PY{p}{,} \PY{n}{dados\PYZus{}metro}\PY{p}{[}\PY{p}{:}\PY{l+m+mi}{720}\PY{p}{]}\PY{o}{.}\PY{n}{traffic\PYZus{}volume}\PY{p}{,} \PY{n}{label}\PY{o}{=}\PY{l+s+s1}{\PYZsq{}}\PY{l+s+s1}{\PYZsh{} passageiros por hora}\PY{l+s+s1}{\PYZsq{}}\PY{p}{)}
\PY{n}{plt}\PY{o}{.}\PY{n}{gcf}\PY{p}{(}\PY{p}{)}\PY{o}{.}\PY{n}{autofmt\PYZus{}xdate}\PY{p}{(}\PY{p}{)}
\PY{n}{plt}\PY{o}{.}\PY{n}{ylabel}\PY{p}{(}\PY{l+s+s1}{\PYZsq{}}\PY{l+s+s1}{\PYZsh{} passageiros}\PY{l+s+s1}{\PYZsq{}}\PY{p}{)}
\PY{n}{plt}\PY{o}{.}\PY{n}{title}\PY{p}{(}\PY{l+s+s1}{\PYZsq{}}\PY{l+s+s1}{Volume de tráfego por hora entre 2 estações de metrô de Minneapolis: primeiros 30 dias da base de dados}\PY{l+s+s1}{\PYZsq{}}\PY{p}{)}
\PY{n}{plt}\PY{o}{.}\PY{n}{show}\PY{p}{(}\PY{p}{)}
\end{Verbatim}
\end{tcolorbox}

    \begin{center}
    \adjustimage{max size={0.9\linewidth}{0.9\paperheight}}{ExercicioComputacionalParte2_files/ExercicioComputacionalParte2_125_0.png}
    \end{center}
    { \hspace*{\fill} \\}
    
    \hypertarget{criauxe7uxe3o-dos-arquivos-de-treino-e-validauxe7uxe3o}{%
\subsubsection{Criação dos arquivos de treino e
validação}\label{criauxe7uxe3o-dos-arquivos-de-treino-e-validauxe7uxe3o}}

    \begin{tcolorbox}[breakable, size=fbox, boxrule=1pt, pad at break*=1mm,colback=cellbackground, colframe=cellborder]
\prompt{In}{incolor}{61}{\boxspacing}
\begin{Verbatim}[commandchars=\\\{\}]
\PY{k}{def} \PY{n+nf}{problema\PYZus{}5\PYZus{}criar\PYZus{}arquivos}\PY{p}{(}\PY{p}{)}\PY{p}{:}
    \PY{n}{dados\PYZus{}metro} \PY{o}{=} \PY{n}{pd}\PY{o}{.}\PY{n}{read\PYZus{}csv}\PY{p}{(}\PY{l+s+s2}{\PYZdq{}}\PY{l+s+s2}{Metro\PYZus{}Interstate\PYZus{}Traffic\PYZus{}Volume.csv}\PY{l+s+s2}{\PYZdq{}}\PY{p}{,} \PY{n}{sep}\PY{o}{=}\PY{l+s+s2}{\PYZdq{}}\PY{l+s+s2}{,}\PY{l+s+s2}{\PYZdq{}}\PY{p}{)}

    \PY{n}{x1} \PY{o}{=} \PY{n}{atrasar}\PY{p}{(}\PY{n}{dados\PYZus{}metro}\PY{o}{.}\PY{n}{traffic\PYZus{}volume}\PY{p}{,} \PY{l+m+mi}{0}\PY{p}{)}
    \PY{n}{x2} \PY{o}{=} \PY{n}{atrasar}\PY{p}{(}\PY{n}{dados\PYZus{}metro}\PY{o}{.}\PY{n}{traffic\PYZus{}volume}\PY{p}{,} \PY{l+m+mi}{24}\PY{p}{)}
    \PY{n}{x3} \PY{o}{=} \PY{n}{atrasar}\PY{p}{(}\PY{n}{dados\PYZus{}metro}\PY{o}{.}\PY{n}{traffic\PYZus{}volume}\PY{p}{,} \PY{l+m+mi}{144}\PY{p}{)}
    \PY{n}{y} \PY{o}{=} \PY{n}{avancar}\PY{p}{(}\PY{n}{dados\PYZus{}metro}\PY{o}{.}\PY{n}{traffic\PYZus{}volume}\PY{p}{,} \PY{l+m+mi}{24}\PY{p}{)}

    \PY{n}{x1} \PY{o}{=} \PY{n}{ajustar}\PY{p}{(}\PY{n}{x1}\PY{p}{,} \PY{l+m+mi}{144}\PY{p}{,} \PY{l+m+mi}{24}\PY{p}{)}
    \PY{n}{x2} \PY{o}{=} \PY{n}{ajustar}\PY{p}{(}\PY{n}{x2}\PY{p}{,} \PY{l+m+mi}{144}\PY{p}{,} \PY{l+m+mi}{24}\PY{p}{)}
    \PY{n}{x3} \PY{o}{=} \PY{n}{ajustar}\PY{p}{(}\PY{n}{x3}\PY{p}{,} \PY{l+m+mi}{144}\PY{p}{,} \PY{l+m+mi}{24}\PY{p}{)}
    \PY{n}{y} \PY{o}{=} \PY{n}{ajustar}\PY{p}{(}\PY{n}{y}\PY{p}{,} \PY{l+m+mi}{144}\PY{p}{,} \PY{l+m+mi}{24}\PY{p}{)}

    \PY{n}{x} \PY{o}{=} \PY{n}{np}\PY{o}{.}\PY{n}{array}\PY{p}{(}\PY{p}{[}\PY{n}{x1}\PY{p}{,} \PY{n}{x2}\PY{p}{,} \PY{n}{x3}\PY{p}{]}\PY{p}{)}\PY{o}{.}\PY{n}{T}

    \PY{n}{criar\PYZus{}arquivo}\PY{p}{(}\PY{l+s+s1}{\PYZsq{}}\PY{l+s+s1}{x5\PYZus{}desnormalizado.csv}\PY{l+s+s1}{\PYZsq{}}\PY{p}{,} \PY{n}{x}\PY{p}{)}
    \PY{n}{criar\PYZus{}arquivo}\PY{p}{(}\PY{l+s+s1}{\PYZsq{}}\PY{l+s+s1}{y5\PYZus{}desnormalizado.csv}\PY{l+s+s1}{\PYZsq{}}\PY{p}{,} \PY{n}{y}\PY{p}{)}

    \PY{n}{minimo} \PY{o}{=} \PY{n+nb}{min}\PY{p}{(}\PY{n}{x}\PY{o}{.}\PY{n}{min}\PY{p}{(}\PY{p}{)}\PY{p}{,} \PY{n+nb}{min}\PY{p}{(}\PY{n}{y}\PY{p}{)}\PY{p}{)}
    \PY{n}{maximo} \PY{o}{=} \PY{n+nb}{max}\PY{p}{(}\PY{n}{x}\PY{o}{.}\PY{n}{max}\PY{p}{(}\PY{p}{)}\PY{p}{,} \PY{n+nb}{max}\PY{p}{(}\PY{n}{y}\PY{p}{)}\PY{p}{)}

    \PY{n}{x\PYZus{}nomalizado} \PY{o}{=} \PY{n}{normalizar}\PY{p}{(}\PY{n}{x}\PY{p}{,} \PY{n}{minimo}\PY{p}{,} \PY{n}{maximo}\PY{p}{)}
    \PY{n}{y\PYZus{}normalizado} \PY{o}{=} \PY{n}{normalizar}\PY{p}{(}\PY{n}{y}\PY{p}{,} \PY{n}{minimo}\PY{p}{,} \PY{n}{maximo}\PY{p}{)}

    \PY{n}{tamanho\PYZus{}amostra\PYZus{}treino} \PY{o}{=} \PY{l+m+mi}{24} \PY{o}{*} \PY{l+m+mi}{30} \PY{o}{*} \PY{l+m+mi}{12} \PY{o}{*} \PY{l+m+mi}{4}

    \PY{n}{xt} \PY{o}{=} \PY{n}{x\PYZus{}nomalizado}\PY{p}{[}\PY{p}{:}\PY{n}{tamanho\PYZus{}amostra\PYZus{}treino}\PY{p}{]}
    \PY{n}{ydt} \PY{o}{=} \PY{n}{y\PYZus{}normalizado}\PY{p}{[}\PY{p}{:}\PY{n}{tamanho\PYZus{}amostra\PYZus{}treino}\PY{p}{]}
    \PY{n}{xt}\PY{p}{,} \PY{n}{ydt} \PY{o}{=} \PY{n}{embaralhar\PYZus{}dados}\PY{p}{(}\PY{n}{xt}\PY{p}{,} \PY{n}{ydt}\PY{p}{)}
    \PY{n}{criar\PYZus{}arquivo}\PY{p}{(}\PY{l+s+s1}{\PYZsq{}}\PY{l+s+s1}{xte5.csv}\PY{l+s+s1}{\PYZsq{}}\PY{p}{,} \PY{n}{xt}\PY{p}{)}
    \PY{n}{criar\PYZus{}arquivo}\PY{p}{(}\PY{l+s+s1}{\PYZsq{}}\PY{l+s+s1}{ydte5.csv}\PY{l+s+s1}{\PYZsq{}}\PY{p}{,} \PY{n}{ydt}\PY{p}{)}

    \PY{n}{criar\PYZus{}arquivo}\PY{p}{(}\PY{l+s+s1}{\PYZsq{}}\PY{l+s+s1}{xte5\PYZus{}desnormalizado.csv}\PY{l+s+s1}{\PYZsq{}}\PY{p}{,} \PY{n}{desnormalizar}\PY{p}{(}\PY{n}{xt}\PY{p}{,} \PY{n}{minimo}\PY{p}{,} \PY{n}{maximo}\PY{p}{)}\PY{o}{.}\PY{n}{astype}\PY{p}{(}\PY{n+nb}{int}\PY{p}{)}\PY{p}{)}
    \PY{n}{criar\PYZus{}arquivo}\PY{p}{(}\PY{l+s+s1}{\PYZsq{}}\PY{l+s+s1}{ydte5\PYZus{}desnormalizado.csv}\PY{l+s+s1}{\PYZsq{}}\PY{p}{,} \PY{n}{desnormalizar}\PY{p}{(}\PY{n}{ydt}\PY{p}{,} \PY{n}{minimo}\PY{p}{,} \PY{n}{maximo}\PY{p}{)}\PY{o}{.}\PY{n}{astype}\PY{p}{(}\PY{n+nb}{int}\PY{p}{)}\PY{p}{)}

    \PY{n}{xv} \PY{o}{=} \PY{n}{x\PYZus{}nomalizado}\PY{p}{[}\PY{n}{tamanho\PYZus{}amostra\PYZus{}treino}\PY{p}{:}\PY{p}{]}
    \PY{n}{ydv} \PY{o}{=} \PY{n}{y\PYZus{}normalizado}\PY{p}{[}\PY{n}{tamanho\PYZus{}amostra\PYZus{}treino}\PY{p}{:}\PY{p}{]}
    \PY{n}{criar\PYZus{}arquivo}\PY{p}{(}\PY{l+s+s1}{\PYZsq{}}\PY{l+s+s1}{xve5.csv}\PY{l+s+s1}{\PYZsq{}}\PY{p}{,} \PY{n}{xv}\PY{p}{)}
    \PY{n}{criar\PYZus{}arquivo}\PY{p}{(}\PY{l+s+s1}{\PYZsq{}}\PY{l+s+s1}{ydve5.csv}\PY{l+s+s1}{\PYZsq{}}\PY{p}{,} \PY{n}{ydv}\PY{p}{)}

    \PY{n}{criar\PYZus{}arquivo}\PY{p}{(}\PY{l+s+s1}{\PYZsq{}}\PY{l+s+s1}{xve5\PYZus{}desnormalizado.csv}\PY{l+s+s1}{\PYZsq{}}\PY{p}{,} \PY{n}{desnormalizar}\PY{p}{(}\PY{n}{xv}\PY{p}{,} \PY{n}{minimo}\PY{p}{,} \PY{n}{maximo}\PY{p}{)}\PY{o}{.}\PY{n}{astype}\PY{p}{(}\PY{n+nb}{int}\PY{p}{)}\PY{p}{)}
    \PY{n}{criar\PYZus{}arquivo}\PY{p}{(}\PY{l+s+s1}{\PYZsq{}}\PY{l+s+s1}{ydve5\PYZus{}desnormalizado.csv}\PY{l+s+s1}{\PYZsq{}}\PY{p}{,} \PY{n}{desnormalizar}\PY{p}{(}\PY{n}{ydv}\PY{p}{,} \PY{n}{minimo}\PY{p}{,} \PY{n}{maximo}\PY{p}{)}\PY{o}{.}\PY{n}{astype}\PY{p}{(}\PY{n+nb}{int}\PY{p}{)}\PY{p}{)}
\end{Verbatim}
\end{tcolorbox}

    \hypertarget{carregamento-dos-dados-de-treino-e-validauxe7uxe3o}{%
\subsubsection{Carregamento dos dados de treino e
validação}\label{carregamento-dos-dados-de-treino-e-validauxe7uxe3o}}

    \begin{tcolorbox}[breakable, size=fbox, boxrule=1pt, pad at break*=1mm,colback=cellbackground, colframe=cellborder]
\prompt{In}{incolor}{62}{\boxspacing}
\begin{Verbatim}[commandchars=\\\{\}]
\PY{n}{xt} \PY{o}{=} \PY{n}{pd}\PY{o}{.}\PY{n}{read\PYZus{}csv}\PY{p}{(}\PY{l+s+s1}{\PYZsq{}}\PY{l+s+s1}{xte5.csv}\PY{l+s+s1}{\PYZsq{}}\PY{p}{)}\PY{o}{.}\PY{n}{values}\PY{o}{.}\PY{n}{tolist}\PY{p}{(}\PY{p}{)}
\PY{n}{ydt} \PY{o}{=} \PY{n}{pd}\PY{o}{.}\PY{n}{read\PYZus{}csv}\PY{p}{(}\PY{l+s+s1}{\PYZsq{}}\PY{l+s+s1}{ydte5.csv}\PY{l+s+s1}{\PYZsq{}}\PY{p}{)}\PY{o}{.}\PY{n}{iloc}\PY{p}{[}\PY{p}{:}\PY{p}{,} \PY{l+m+mi}{0}\PY{p}{]}\PY{o}{.}\PY{n}{tolist}\PY{p}{(}\PY{p}{)}
\PY{n}{xv} \PY{o}{=} \PY{n}{pd}\PY{o}{.}\PY{n}{read\PYZus{}csv}\PY{p}{(}\PY{l+s+s1}{\PYZsq{}}\PY{l+s+s1}{xve5.csv}\PY{l+s+s1}{\PYZsq{}}\PY{p}{)}\PY{o}{.}\PY{n}{values}\PY{o}{.}\PY{n}{tolist}\PY{p}{(}\PY{p}{)}
\PY{n}{ydv} \PY{o}{=} \PY{n}{pd}\PY{o}{.}\PY{n}{read\PYZus{}csv}\PY{p}{(}\PY{l+s+s1}{\PYZsq{}}\PY{l+s+s1}{ydve5.csv}\PY{l+s+s1}{\PYZsq{}}\PY{p}{)}\PY{o}{.}\PY{n}{iloc}\PY{p}{[}\PY{p}{:}\PY{p}{,} \PY{l+m+mi}{0}\PY{p}{]}\PY{o}{.}\PY{n}{tolist}\PY{p}{(}\PY{p}{)}

\PY{n}{x} \PY{o}{=} \PY{n}{pd}\PY{o}{.}\PY{n}{read\PYZus{}csv}\PY{p}{(}\PY{l+s+s1}{\PYZsq{}}\PY{l+s+s1}{x5\PYZus{}desnormalizado.csv}\PY{l+s+s1}{\PYZsq{}}\PY{p}{)}\PY{o}{.}\PY{n}{values}
\PY{n}{y} \PY{o}{=} \PY{n}{pd}\PY{o}{.}\PY{n}{read\PYZus{}csv}\PY{p}{(}\PY{l+s+s1}{\PYZsq{}}\PY{l+s+s1}{y5\PYZus{}desnormalizado.csv}\PY{l+s+s1}{\PYZsq{}}\PY{p}{)}\PY{o}{.}\PY{n}{values}
\PY{n}{minimo} \PY{o}{=} \PY{n+nb}{min}\PY{p}{(}\PY{n}{x}\PY{o}{.}\PY{n}{min}\PY{p}{(}\PY{p}{)}\PY{p}{,} \PY{n}{y}\PY{o}{.}\PY{n}{min}\PY{p}{(}\PY{p}{)}\PY{p}{)}
\PY{n}{maximo} \PY{o}{=} \PY{n+nb}{max}\PY{p}{(}\PY{n}{x}\PY{o}{.}\PY{n}{max}\PY{p}{(}\PY{p}{)}\PY{p}{,} \PY{n}{y}\PY{o}{.}\PY{n}{max}\PY{p}{(}\PY{p}{)}\PY{p}{)}
\end{Verbatim}
\end{tcolorbox}

    \hypertarget{resultado-desejado}{%
\subsubsection{Resultado desejado}\label{resultado-desejado}}

    \begin{tcolorbox}[breakable, size=fbox, boxrule=1pt, pad at break*=1mm,colback=cellbackground, colframe=cellborder]
\prompt{In}{incolor}{63}{\boxspacing}
\begin{Verbatim}[commandchars=\\\{\}]
\PY{n}{label\PYZus{}y\PYZus{}validacao} \PY{o}{=} \PY{l+s+s2}{\PYZdq{}}\PY{l+s+s2}{y(k+24) = g[y(k),y(k\PYZhy{}24),y(k\PYZhy{}144)]}\PY{l+s+s2}{\PYZdq{}}
\PY{n}{exibir\PYZus{}resultado\PYZus{}desejado}\PY{p}{(}\PY{n}{xv}\PY{p}{,} \PY{n}{ydv}\PY{p}{,} \PY{n}{label\PYZus{}y\PYZus{}validacao}\PY{p}{)}
\end{Verbatim}
\end{tcolorbox}

    \begin{center}
    \adjustimage{max size={0.9\linewidth}{0.9\paperheight}}{ExercicioComputacionalParte2_files/ExercicioComputacionalParte2_131_0.png}
    \end{center}
    { \hspace*{\fill} \\}
    
    \hypertarget{resumo-dos-dados-de-treino-e-validauxe7uxe3o}{%
\subsubsection{Resumo dos dados de treino e
validação}\label{resumo-dos-dados-de-treino-e-validauxe7uxe3o}}

    \begin{tcolorbox}[breakable, size=fbox, boxrule=1pt, pad at break*=1mm,colback=cellbackground, colframe=cellborder]
\prompt{In}{incolor}{64}{\boxspacing}
\begin{Verbatim}[commandchars=\\\{\}]
\PY{n}{descrever\PYZus{}dados}\PY{p}{(}\PY{n}{xt}\PY{p}{,} \PY{n}{ydt}\PY{p}{,} \PY{n}{xv}\PY{p}{,} \PY{n}{ydv}\PY{p}{)}
\end{Verbatim}
\end{tcolorbox}

    \begin{Verbatim}[commandchars=\\\{\}]

===Dataset xt===
              0         1         2
0      0.905220  0.639286  0.615934
1      0.409753  0.382967  0.657692
2      0.482418  0.082692  0.624725
3      0.718407  0.202198  0.326786
4      0.684890  0.739835  0.239148
{\ldots}         {\ldots}       {\ldots}       {\ldots}
34554  0.101236  0.169093  0.707005
34555  0.037500  0.221566  0.592995
34556  0.474451  0.045604  0.102885
34557  0.047115  0.056319  0.850549
34558  0.469780  0.347940  0.040247

[34559 rows x 3 columns]

===Dataset yt===
              0
0      0.328846
1      0.359478
2      0.441621
3      0.108654
4      0.320604
{\ldots}         {\ldots}
34554  0.745055
34555  0.779945
34556  0.657967
34557  0.048214
34558  0.436538

[34559 rows x 1 columns]

===Dataset xv===
              0         1         2
0      0.809066  0.778159  0.802747
1      0.903709  0.826786  0.623901
2      0.830632  0.691346  0.486401
3      0.685027  0.691346  0.472390
4      0.611264  0.607692  0.454945
{\ldots}         {\ldots}       {\ldots}       {\ldots}
13470  0.405220  0.372665  0.597665
13471  0.358104  0.486813  0.613736
13472  0.529670  0.238049  0.622390
13473  0.250824  0.119093  0.608929
13474  0.126374  0.071978  0.661538

[13475 rows x 3 columns]

===Dataset yv===
              0
0      0.395604
1      0.767170
2      0.826648
3      0.826648
4      0.830220
{\ldots}         {\ldots}
13470  0.486676
13471  0.382005
13472  0.296566
13473  0.199176
13474  0.131044

[13475 rows x 1 columns]
    \end{Verbatim}

    \hypertarget{estruturas-implementadas}{%
\subsubsection{Estruturas
implementadas}\label{estruturas-implementadas}}

\hypertarget{paruxe2metros-gerais}{%
\paragraph{Parâmetros gerais}\label{paruxe2metros-gerais}}

    \begin{tcolorbox}[breakable, size=fbox, boxrule=1pt, pad at break*=1mm,colback=cellbackground, colframe=cellborder]
\prompt{In}{incolor}{65}{\boxspacing}
\begin{Verbatim}[commandchars=\\\{\}]
\PY{n}{n\PYZus{}epoca} \PY{o}{=} \PY{l+m+mi}{20}
\end{Verbatim}
\end{tcolorbox}

    \hypertarget{estrutura-anfis}{%
\paragraph{Estrutura ANFIS}\label{estrutura-anfis}}

Parâmetros

    \begin{tcolorbox}[breakable, size=fbox, boxrule=1pt, pad at break*=1mm,colback=cellbackground, colframe=cellborder]
\prompt{In}{incolor}{66}{\boxspacing}
\begin{Verbatim}[commandchars=\\\{\}]
\PY{n}{m} \PY{o}{=} \PY{l+m+mi}{12}
\PY{n}{alfa} \PY{o}{=} \PY{l+m+mf}{0.1}
\PY{n}{anfis\PYZus{}problema\PYZus{}5} \PY{o}{=} \PY{n}{ANFIS}\PY{p}{(}\PY{n}{n\PYZus{}epoca}\PY{p}{,} \PY{n}{xt}\PY{p}{,} \PY{n}{ydt}\PY{p}{,} \PY{n}{xv}\PY{p}{,} \PY{n}{ydv}\PY{p}{,} \PY{n}{m}\PY{p}{,} \PY{n}{label\PYZus{}y\PYZus{}validacao}\PY{p}{,} \PY{n}{alfa}\PY{p}{)}
\end{Verbatim}
\end{tcolorbox}

    Demonstração de estado inicial

    \begin{tcolorbox}[breakable, size=fbox, boxrule=1pt, pad at break*=1mm,colback=cellbackground, colframe=cellborder]
\prompt{In}{incolor}{67}{\boxspacing}
\begin{Verbatim}[commandchars=\\\{\}]
\PY{n}{anfis\PYZus{}problema\PYZus{}5}\PY{o}{.}\PY{n}{exibir\PYZus{}resultado\PYZus{}validacao}\PY{p}{(}\PY{k+kc}{True}\PY{p}{,} \PY{n}{minimo}\PY{p}{,} \PY{n}{maximo}\PY{p}{)}
\end{Verbatim}
\end{tcolorbox}

    \begin{center}
    \adjustimage{max size={0.9\linewidth}{0.9\paperheight}}{ExercicioComputacionalParte2_files/ExercicioComputacionalParte2_139_0.png}
    \end{center}
    { \hspace*{\fill} \\}
    
    \begin{Verbatim}[commandchars=\\\{\}]
Root mean squared error (RMSE):  3466.6835
Mean absolute percentage error (MAPE): 330.3883\%
    \end{Verbatim}

    Demonstração de treinamento

    \begin{tcolorbox}[breakable, size=fbox, boxrule=1pt, pad at break*=1mm,colback=cellbackground, colframe=cellborder]
\prompt{In}{incolor}{68}{\boxspacing}
\begin{Verbatim}[commandchars=\\\{\}]
\PY{n}{anfis\PYZus{}problema\PYZus{}5}\PY{o}{.}\PY{n}{treinar\PYZus{}gradiente}\PY{p}{(}\PY{n}{plota\PYZus{}resultado\PYZus{}epocas}\PY{o}{=}\PY{k+kc}{True}\PY{p}{,}
                                   \PY{n}{desnormalizar\PYZus{}apresentacao\PYZus{}resultado}\PY{o}{=}\PY{k+kc}{True}\PY{p}{,}
                                   \PY{n}{minimo\PYZus{}desnomalizacao}\PY{o}{=}\PY{n}{minimo}\PY{p}{,}
                                   \PY{n}{maximo\PYZus{}desnomalizacao}\PY{o}{=}\PY{n}{maximo}\PY{p}{)}
\end{Verbatim}
\end{tcolorbox}

    \begin{center}
    \adjustimage{max size={0.9\linewidth}{0.9\paperheight}}{ExercicioComputacionalParte2_files/ExercicioComputacionalParte2_141_0.png}
    \end{center}
    { \hspace*{\fill} \\}
    
    \begin{center}
    \adjustimage{max size={0.9\linewidth}{0.9\paperheight}}{ExercicioComputacionalParte2_files/ExercicioComputacionalParte2_141_1.png}
    \end{center}
    { \hspace*{\fill} \\}
    
    Demonstração de resultado final

    \begin{tcolorbox}[breakable, size=fbox, boxrule=1pt, pad at break*=1mm,colback=cellbackground, colframe=cellborder]
\prompt{In}{incolor}{69}{\boxspacing}
\begin{Verbatim}[commandchars=\\\{\}]
\PY{n}{anfis\PYZus{}problema\PYZus{}5}\PY{o}{.}\PY{n}{exibir\PYZus{}resultado\PYZus{}validacao}\PY{p}{(}\PY{k+kc}{True}\PY{p}{,} \PY{n}{minimo}\PY{p}{,} \PY{n}{maximo}\PY{p}{)}
\end{Verbatim}
\end{tcolorbox}

    \begin{center}
    \adjustimage{max size={0.9\linewidth}{0.9\paperheight}}{ExercicioComputacionalParte2_files/ExercicioComputacionalParte2_143_0.png}
    \end{center}
    { \hspace*{\fill} \\}
    
    \begin{Verbatim}[commandchars=\\\{\}]
Root mean squared error (RMSE):  1823.9816
Mean absolute percentage error (MAPE): 136.2210\%
    \end{Verbatim}

    Resultado médio para 10 treinos

    \begin{tcolorbox}[breakable, size=fbox, boxrule=1pt, pad at break*=1mm,colback=cellbackground, colframe=cellborder]
\prompt{In}{incolor}{70}{\boxspacing}
\begin{Verbatim}[commandchars=\\\{\}]
\PY{n}{anfis\PYZus{}problema\PYZus{}5}\PY{o}{.}\PY{n}{exibir\PYZus{}resultado\PYZus{}validacao\PYZus{}multiplos\PYZus{}treinos}\PY{p}{(}\PY{l+m+mi}{10}\PY{p}{,} \PY{k+kc}{True}\PY{p}{,} \PY{n}{minimo}\PY{p}{,} \PY{n}{maximo}\PY{p}{)}
\end{Verbatim}
\end{tcolorbox}

    \begin{center}
    \adjustimage{max size={0.9\linewidth}{0.9\paperheight}}{ExercicioComputacionalParte2_files/ExercicioComputacionalParte2_145_0.png}
    \end{center}
    { \hspace*{\fill} \\}
    
    \begin{Verbatim}[commandchars=\\\{\}]
Média do RMSE:  1822.0256
Desvio padrão do RMSE:  1.4384
    \end{Verbatim}

    \begin{center}
    \adjustimage{max size={0.9\linewidth}{0.9\paperheight}}{ExercicioComputacionalParte2_files/ExercicioComputacionalParte2_145_2.png}
    \end{center}
    { \hspace*{\fill} \\}
    
    \begin{Verbatim}[commandchars=\\\{\}]
Média do MAPE: 136.5400\%
Desvio padrão do MAPE: 0.3504\%
    \end{Verbatim}

    \hypertarget{estrutura-nfn}{%
\subsubsection{Estrutura NFN}\label{estrutura-nfn}}

Parâmetros

    \begin{tcolorbox}[breakable, size=fbox, boxrule=1pt, pad at break*=1mm,colback=cellbackground, colframe=cellborder]
\prompt{In}{incolor}{71}{\boxspacing}
\begin{Verbatim}[commandchars=\\\{\}]
\PY{n}{m} \PY{o}{=} \PY{l+m+mi}{12}
\PY{n}{alfa} \PY{o}{=} \PY{l+m+mf}{0.1}

\PY{n}{nfn\PYZus{}problema\PYZus{}5} \PY{o}{=} \PY{n}{NFN}\PY{p}{(}\PY{n}{n\PYZus{}epoca}\PY{p}{,} \PY{n}{xt}\PY{p}{,} \PY{n}{ydt}\PY{p}{,} \PY{n}{xv}\PY{p}{,} \PY{n}{ydv}\PY{p}{,} \PY{n}{m}\PY{p}{,} \PY{n}{label\PYZus{}y\PYZus{}validacao}\PY{p}{,} \PY{n}{alfa}\PY{p}{)}
\end{Verbatim}
\end{tcolorbox}

    Demonstração do estado inicial

    \begin{tcolorbox}[breakable, size=fbox, boxrule=1pt, pad at break*=1mm,colback=cellbackground, colframe=cellborder]
\prompt{In}{incolor}{72}{\boxspacing}
\begin{Verbatim}[commandchars=\\\{\}]
\PY{n}{nfn\PYZus{}problema\PYZus{}5}\PY{o}{.}\PY{n}{exibir\PYZus{}resultado\PYZus{}validacao}\PY{p}{(}\PY{k+kc}{True}\PY{p}{,} \PY{n}{minimo}\PY{p}{,} \PY{n}{maximo}\PY{p}{)}
\end{Verbatim}
\end{tcolorbox}

    \begin{center}
    \adjustimage{max size={0.9\linewidth}{0.9\paperheight}}{ExercicioComputacionalParte2_files/ExercicioComputacionalParte2_149_0.png}
    \end{center}
    { \hspace*{\fill} \\}
    
    \begin{Verbatim}[commandchars=\\\{\}]
Root mean squared error (RMSE):  8597.779
Mean absolute percentage error (MAPE): 671.2134\%
    \end{Verbatim}

    Demonstração de treinamento

    \begin{tcolorbox}[breakable, size=fbox, boxrule=1pt, pad at break*=1mm,colback=cellbackground, colframe=cellborder]
\prompt{In}{incolor}{73}{\boxspacing}
\begin{Verbatim}[commandchars=\\\{\}]
\PY{n}{nfn\PYZus{}problema\PYZus{}5}\PY{o}{.}\PY{n}{treinar\PYZus{}gradiente}\PY{p}{(}\PY{n}{plota\PYZus{}resultado\PYZus{}epocas}\PY{o}{=}\PY{k+kc}{True}\PY{p}{,}
                                 \PY{n}{desnormalizar\PYZus{}apresentacao\PYZus{}resultado}\PY{o}{=}\PY{k+kc}{True}\PY{p}{,}
                                 \PY{n}{minimo\PYZus{}desnomalizacao}\PY{o}{=}\PY{n}{minimo}\PY{p}{,}
                                 \PY{n}{maximo\PYZus{}desnomalizacao}\PY{o}{=}\PY{n}{maximo}\PY{p}{)}
\end{Verbatim}
\end{tcolorbox}

    \begin{center}
    \adjustimage{max size={0.9\linewidth}{0.9\paperheight}}{ExercicioComputacionalParte2_files/ExercicioComputacionalParte2_151_0.png}
    \end{center}
    { \hspace*{\fill} \\}
    
    \begin{center}
    \adjustimage{max size={0.9\linewidth}{0.9\paperheight}}{ExercicioComputacionalParte2_files/ExercicioComputacionalParte2_151_1.png}
    \end{center}
    { \hspace*{\fill} \\}
    
    Demonstração de resultado final

    \begin{tcolorbox}[breakable, size=fbox, boxrule=1pt, pad at break*=1mm,colback=cellbackground, colframe=cellborder]
\prompt{In}{incolor}{74}{\boxspacing}
\begin{Verbatim}[commandchars=\\\{\}]
\PY{n}{nfn\PYZus{}problema\PYZus{}5}\PY{o}{.}\PY{n}{exibir\PYZus{}resultado\PYZus{}validacao}\PY{p}{(}\PY{k+kc}{True}\PY{p}{,} \PY{n}{minimo}\PY{p}{,} \PY{n}{maximo}\PY{p}{)}
\end{Verbatim}
\end{tcolorbox}

    \begin{center}
    \adjustimage{max size={0.9\linewidth}{0.9\paperheight}}{ExercicioComputacionalParte2_files/ExercicioComputacionalParte2_153_0.png}
    \end{center}
    { \hspace*{\fill} \\}
    
    \begin{Verbatim}[commandchars=\\\{\}]
Root mean squared error (RMSE):  1935.8699
Mean absolute percentage error (MAPE): 134.6726\%
    \end{Verbatim}

    Resultado médio para 10 treinos

    \begin{tcolorbox}[breakable, size=fbox, boxrule=1pt, pad at break*=1mm,colback=cellbackground, colframe=cellborder]
\prompt{In}{incolor}{75}{\boxspacing}
\begin{Verbatim}[commandchars=\\\{\}]
\PY{n}{nfn\PYZus{}problema\PYZus{}5}\PY{o}{.}\PY{n}{exibir\PYZus{}resultado\PYZus{}validacao\PYZus{}multiplos\PYZus{}treinos}\PY{p}{(}\PY{n}{numero\PYZus{}treinos}\PY{o}{=}\PY{l+m+mi}{10}\PY{p}{,} \PY{n}{desnormalizar\PYZus{}resultado}\PY{o}{=}\PY{k+kc}{True}\PY{p}{,} \PY{n}{minimo\PYZus{}desnomalizacao}\PY{o}{=}\PY{n}{minimo}\PY{p}{,} \PY{n}{maximo\PYZus{}desnomalizacao}\PY{o}{=}\PY{n}{maximo}\PY{p}{)}
\end{Verbatim}
\end{tcolorbox}

    \begin{center}
    \adjustimage{max size={0.9\linewidth}{0.9\paperheight}}{ExercicioComputacionalParte2_files/ExercicioComputacionalParte2_155_0.png}
    \end{center}
    { \hspace*{\fill} \\}
    
    \begin{Verbatim}[commandchars=\\\{\}]
Média do RMSE:  1935.8699
Desvio padrão do RMSE:  0.0
    \end{Verbatim}

    \begin{center}
    \adjustimage{max size={0.9\linewidth}{0.9\paperheight}}{ExercicioComputacionalParte2_files/ExercicioComputacionalParte2_155_2.png}
    \end{center}
    { \hspace*{\fill} \\}
    
    \begin{Verbatim}[commandchars=\\\{\}]
Média do MAPE: 134.6726\%
Desvio padrão do MAPE: 0.0000\%
    \end{Verbatim}

    \hypertarget{referuxeancias}{%
\subsection{Referências}\label{referuxeancias}}

Chiu, Stephen L. ``Fuzzy model identification based on cluster
estimation.'' Journal of Intelligent \& fuzzy systems 2.3 (1994):
267-278.

Ghosh, S., and Dubey, S. K. . ``Comparative analysis of k-means and
fuzzy c-means algorithms.'' International Journal of Advanced Computer
Science and Applications 4.4 (2013).


    % Add a bibliography block to the postdoc
    
    
    
\end{document}
